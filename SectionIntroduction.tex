\section{Introduction}
Many open problems in star formation, molecular cloud evolution, and galaxy-scale gas dynamics remain open because it has not been possible to measure the most useful quantity for resolving them -- the vector field of 3D gas velocity and density over an extended area of sky. A measurement of this field would allow us to solve the continuity equation \cite{euler1757principes}, and measure the rate at which density is increasing in any region of the Galaxy, and over what physical scale. 

Flows of converging gas are a central part of theories of the formation of giant molecular clouds (GMCs) in spiral arms \cite{Vazquez_Semadeni_2007, Audit_2005}. Suggested mechanisms for forming GMCs out of diffuse gas include different sorts of thermodynamic, hydrodynamic, and magnetohydrodynamic instabilities, particularly in colliding flows \citep{Clark:2012bq,2014ApJ...790...37C, Heitsch06}; gas compression driven by large-scale structures such as spiral arms and expanding supergiant shells \citep{Roberts:1972bp,Bonnell:2006hn,Fujimoto:2014kh}; and collapse due to self-gravity \citep{Kim:2002da,2012MNRAS.425.2157D,VazquezSemadeni:2007cj}.
There is another set of mechanisms in which small, dense cloudlets form by one of the possibilities listed above and then grow, by cloudlet-cloudlet agglomeration \citep{Roberts:1987eb,Dobbs:2008ez,Tasker:2009gc} or diffuse ISM accretion \citep{Goldbaum:2011kj,Heitsch:2013jp}, into GMCs. It is very hard to distinguish observationally between these mechanisms using the standard observable of molecular clouds, emission maps from molecular tracers like carbon monoxide (CO). A tool that could measure both the distance and the velocity of these flows could make significant progress. 

Gas flows on much larger scales, driven by spiral arms, may dramatically effect the pattern of galaxy-wide star formation \cite{Roberts_1972, Bonnell_2006}. These spiral shocks have been seen in strongly tidally interacting two-arm spiral galaxies in the nearby universe \cite{Visser:1980vc, Visser:1980ud, Shetty_2007} using carbon monoxide (CO) and neutral hydrogen (HI) observations. Unfortunately, the resolution in HI required to test these theories, beyond very nearby galaxies with extreme two-armed spiral structure, is not yet observationally feasible \cite(Visser:1980ud). Thus on larger scales we also need new methods to answer critical questions about how galaxies form molecular clouds and stars. 

Since non-line-of-sight velocities are derived from proper motions, which cannot be reliably measured for extended sources, 3-velocity fields are likely to remain inaccessible. 
For many of these open problems, a measurement of the line-of-sight velocity as a function of 3D position, the 1-velocity field, would be enough to make non-trivial progress. 
We have developed a method for deriving 1-velocity fields of the interstellar medium in the Milky Way from position-position-velocity (PPV) and position-position-position (PPP) measurements. 

This method can be understood as a variation on CfR's extension of kinematic distances.
The kinematic distance to a point at some on-sky position ($\ell$, b) and line-of-sight velocity $v$ is the distance at which a model of gas motion in that ($\ell$, b) direction is equal to $v$.
The gas motion model is usually taken to be a galactic rotation curve. 

Kinematic distances have several well-known drawbacks.
Firstly, they are not unique in directions where different distances are assigned the same line-of-sight velocity.
When a flat rotation curve is assumed, sightlines inside the solar circle and towards the galactic anti-center do not have unique distances. 
Next, kinematic distances ignore the possibility of deviations from the gas motion model.
Measurements of the distances and line-of-sight velocities of point-like objects such as masers show that deviations from azimuthally symmetric rotation, or streaming motions, are the norm rather than the exception.
In some cases, streaming motions can mean that there is no distance corresponding to a given velocity. 
This is known to happen towards the galactic anti-center CITE and near the galactic center CITE.
Finally, computing kinematic distances for an entire PPV dataset yields a PPP dataset, trading one map of three-dimensional structure for another.
Despite these drawbacks, kinematic distances are an adequate estimate of actual distances over much of the galactic plane.

CfR's method improves on kinematic distances by treating them as one of a number of context clues to the distance to an object.
Other context clues include the physical size and height above the galactic plane that a distance would imply given the object's angular size and height above the galactic plane, the presence or absence of H I 21cm self-absorption in the direction of the object, and the distances to any other objects that are known to be close to the object of interest.
If there are enough context clues, this approach can help resolve the degeneracy between distances that correspond to the same line-of-sight velocity and can be used to detect and measure streaming motions.

If we think of kinematic distances as a somewhat unstable zeroth-order approximation to the distance corresponding to a point in a PPV dataset, CfR's method is a second-order correction with improved stability properties.
From this perspective, building on CfR's method is unlikely to lead to  significant improvements to their solution.
Instead, we apply a rudimentary form of their approach to the complementary problem of assigning a velocity to a point in a PPP dataset.

With the advent of large optical and near-infrared photometric surveys, it has become possible to build maps of the three-dimensional distribution of dust out several kiloparsecs from the Sun CITE.
If we make some assumptions about the mixing and relative abundances of dust and gas in the ISM, we can consider these maps to be of the three-dimensional stucture of the ISM. 

To get a zeroth-order velocity for a point in one of these PPP maps, we can take inspiration from kinematic distances and evaluate the galactic rotation curve at its position. 
Even at this stage, the PPP to PPPV problem has an advantage over the PPV to PPPV problem -- the position to velocity assignment is unique, if not necessarily accurate.
There are different possible approaches to improving this approximation.
Our approach is to use a PPV dataset of the ISM over the same area as a guide.
Reproducing this PPV dataset requires modestly adjusting the velocities assigned to the points in the PPP dataset.
These adjusted velocity assignments are our PPPV map.

We have checked this map against the maser measurements in CITET REID (R14). 
Reid et al. have measured the distance to and line-of-sight velocity of a number of masers.
Our PPPV map evaluated at the on-sky positions and distances of the masers agrees with the masers' measured line-of-sight velocities.
This is particularly heartening considering that many of the masers are known to deviate significantly from galactic rotation.

In this work, we describe our method, the map it produces, and our evaluation of this map's accuracy and precision.
In Section \ref{sec:data}, we describe the datasets we use to make and evaluate our PPPV map.
In Section \ref{sec:methods}, we give a detailed explanation of the PPPV mapping technique and a brief overview of how to fit a rotation curve to a PPPV map.
In Section \ref{sec:results}, we present the PPPV map and its corresponding rotation curve.
In Section \ref{sec:validation}, we compare the PPPV map to earlier work and discuss the probable effects of known systematics.
In Section \ref{sec:conclusion}, we conclude.