\section{Discussion}
\label{sec:discussion}

\subsection{Ways in which our assumptions can fail}
\label{sec:discussion-systematics}
In our description of kinetic tomography, we make three main assumptions -- we can linearly scale data cubes of gas and dust tracers to get accurate PPV and PPP maps of the ISM, the distribution of velocities associated with a single PPP voxel can be well-approximated by a narrow Gaussian, and insisting that the central velocities of neighboring PPP voxels should be similar regularizes the PPPV solution in a valid and useful way.
We have not made the claim that kinetic tomography will be in any sense \emph{optimal} (CITE DUSTIN'S APRIL FOOLS PAPER) if these assumptions are true, but we have made the claim that kinetic tomography will be \emph{pretty good}. 
To what extent do these assumptions appear to be true of the data we have to work with and how does their failure affect the kinetic tomography solution?

The first assumption, that linearly scaling data cubes of gas and dust tracers yields accurate PPV and PPP maps of the ISM, has several localized failure modes and two global failure modes. 
The local failure modes are due to observational and astrophysical phenomena, such as $\HI$ self-absorption and variations in the dust reddening law, which produce localized non-linearities and variations in the conversion from tracer to mass. 
The global failure modes arise from the maximum distance mapped by the dust tracer cube being shorter than the maximum distance mapped by the gas tracer cubes. 
By construction, the gas tracers cubes map all gas within a certain velocity range. 
Gas within this velocity range can be arbitrarily far away. 
As a result, the gas tracer cubes map out the entire galaxy.
The dust tracer cube is built by analyzing reddening towards stars. If there are no measurements of stars beyond some distance, the dust tracer cube cuts off.
This cutoff can happen due to large reddenings (e.g. on the far side of the Galactic center) or the spatial structure of the Milky Way's stellar population (e.g. at large heights above the Galactic plane).

We can get an idea of the magnitude and extent of these different PPV-PPP cube mistmatches by comparing the projections of the PPV and PPP cubes to the $\glon$-$\glat$ plane. 
The distribution of gas-derived versus reddening-derived total columns is shown in figure \ref{fig:tracer_syst_values}. 
The two sets of total columns are mostly consistent with each other, i.e. lie along the orange line. 
There are two regimes in which the total columns are not consistent -- at gas columns below $10^{21}$ cm$^{-2}$ (marked in  blue) and at gas columns above CHECK THE NUMBER $3 \times 10^{22}$ cm$^{-2}$ (marked in red). 
In both cases, the reddening-derived total column is noticeably smaller than the gas-derived total column.

The spatial distribution of sightlines in the blue and red areas is shown in figure \ref{fig:tracer_syst_spatial} along with the on-sky positions of the HMSFRs from section \ref{sec:KT-validation}. 
The low-column sightlines with less reddening-traced than gas-traced total column, which are marked in blue, all contain gas at large heights above the Galactic plane. 
This gas is not well-traced by reddening because (1) there are fewer stars at large heights above the Galactic plane, limiting the maximum distance accurately described by the PPP cube at high latitude; and (2) the dust-to-gas ratio at large heights above the Galactic plane is thought to be lower than it is in the Galactic plane CITATION. 
Since we only use sightlines with $\glat$ between $-10^\circ$ and $+10^\circ$ and most of the low-column under-reddened gas is outside of this region, we do not currently need to worry about this effect.

The high-column sightlines with less reddening-traced than gas-traced total column are marked in red.
These sightlines pass through dense parts of molecular clouds and/or the Galactic plane within the solar circle. 
In both cases, the PPP cube underestimates the total column because a high-reddening feature nearby, be it the galactic center or the dense molecular cloud, blocks out more distant stars. 
Since the gas tracers are measured in emission rather than absorption, the PPV cube includes the more distant parts of the ISM which are missing from the PPP cube. 
Secondary effects, such as variations in dust absorption properties in dense environments, may also play a role.


The second assumption, that the distribution of velocities associated with a single voxel in the PPP cube can be accurately approximated by a narrow Gaussian, is intuitively wrong but works well-enough in practice for much of the volume covered by the PPP cube.

WHAT'S THE ARGUMENT HERE? ASSUME THAT THIS IS BASICALLY TRUE FOR A VOXEL WHICH IS "SMALL ENOUGH" IN PHYSICAL TERMS, I.E. IN DX, DY, DZ RATHER THAN DL, DB, DMU. HOW BIG CAN THIS VOXEL BE? 

TWO LIMITING CASES -- VOXEL IS SMALL ENOUGH THAT THE ASSUMPTION IS VERY GOOD, VOXEL IS BIG ENOUGH THAT THE V DISTRIBUTION CONTAINS MULTIPLE BIG PEAKS. HOW MANY DISTINCT VELOCITY THINGS ARE THERE IN A PARSEC-DEEP THING?

TWO MORE LIMITING CASES -- THE ISM AS A BUNCH OF CLUMPS WHICH MOVE A LOT RELATIVE TO EACH OTHER VS. THE ISM AS A THING WHOSE VELOCITY FIELD VARIES PRETTY SMOOTHLY FOR THE MOST PART EXCEPT FOR SOME SMALL PLACES. 

LET'S THINK ABOUT HOW THESE TWO CONCEPTS -- VOXEL SIZE AND ISM VELOCITY STRUCTURE SCALE -- INTERACT. 
TAKE THE JITTERY ISM CASE. A VOXEL WHICH IS "SMALL ENOUGH" CONTAINS ONE ``LUMP''. A VOXEL BECOMES TOO BIG WHEN IT CONTAINS MULTIPLE LUMPS WITH DIFFERENT ENOUGH VELOCITIES THAT THEIR VELOCITY GAUSSIANS DON'T ADD UP TO ~ANOTHER GAUSSIAN. SO, THEIR VELOCITIES SHOULD DIFFER BY MORE THAN SOME FRACTION OF EACH LUMP'S INTERNAL DISPERSION. 

TAKE THE SMOOTH ISM CASE. IF THE MASS DISTRIBUTION IS COMPARABLY SMOOTH, THEN THE VELOCITY DISTRIBUTION OF THE VOXEL IS A CONVOLUTION OF A GAUSSIAN WITH A WIGGLY TOPHAT. IF THE TOPHAT ISN'T ALL THAT MUCH WIDER COMPARED TO THE GAUSSIAN, I.E. IF THE CHANGE IN ROTATION CURVE + LOW FREQUENCY STREAMING MOTIONS ACROSS A VOXEL IS SOME FRACTION OF THE ``SMALL VOXEL'' VELOCITY DISPERSION, THEN THINGS ARE FINE. 


The third assumption is that our regularization scheme, insisting that neighboring pixels have similar 



Despite all of the ways in which our assumptions are clearly not always true, kinetic tomography still produces \emph{pretty good} results in the regimes we can currently test (see sections \ref{sec:KT-validation} and \ref{sec:rotation_curve}). 
The kinetic tomography solution has not yet been tested outside of the two limiting cases of the detailed velocities of the environs of HMSFRs and the azimuthally averaged velocities of most of the near half of the galaxy.
Studies attempting to apply kinetic tomography outside of those two rather narrowly defined cases will need to carefully demonstrate the robustness of their conclusions to the lengthy list of likely sources of systematic error given above.

\subsection{Catastrophic outliers}
\label{sec:discussion-catastrophic}
FOR WHATEVER REASON, THE BAD MASERS JUST REALLY DON'T LOOK VERY DIFFERENT FROM THE GOOD MASERS. 

THE ONE THING, WHICH IS HARD TO QUANTIFY BUT MAAAAAYBE SORT OF THERE IS THAT THE OUTLIERS' VELOCITIES ARE PRETTY DIFFERENT FROM THE VELOCITIES OF THEIR NEIGHBORS? 

HOW CAN WE TRY TO QUANTIFY THIS -- PICK A NUMBER OF NEAREST NEIGHBORS (SELECTED IN 3D), LOOK AT DISTRIBUTION OF STANDARD DEV OF THOSE NEIGHBORS' VELOCITIES, SEE WHERE THE OUTLIERS FALL IN IT

IF, AND THAT'S STILL AN IF, "THESE ARE DIFFERENT FROM THEIR NEIGHBORS AND THEIR NEIGHBORS LOOK LIKE THE GAS" ENDS UP BEING TRUE, HOW CAN WE INTERPRET THAT? WELL, MAYBE THE OUTLIERS ARE THE ONES WHAT GOT KICKED IN THE PROCESS OF FORMATION, SO THEIR KINEMATICS HAVE DECOUPLED FROM THE GAS.

AND ANOTHER THING -- ARE THESE SOURCES SYSTEMATICALLY _OLDER_? BECAUSE YOU'D KIND OF EXPECT THE OLD STUFF TO HAVE DECOUPLED MORE FROM THE GAS, RIGHT? MAYBE THERE'S A DIFFERENCE IN TERMS OF MASER TYPE, AS WELL?

\subsection{The surprising (?) strength of large-scale streaming motions}
THIS CAN BE A COMMENT ABOUT THE EXTREME ~MAGNITUDE~ OF SPATIALLY DEVIATIONS FROM CIRCULAR ROTATION, SINCE THAT SEEMED TO BE A COMMENT-WORTHY THING.
BASICALLY SOMETHING ALONG THE LINES OF "STUFF IS MOVING AROUND A BUNCH, AND EVEN IF WE DON'T GET ~HOW~ IT'S MOVING AROUND EXACTLY RIGHT OVER THE STUFF'S FULL EXTENT."

WHY IS THIS INTERESTING? WELLLLLL, FIRST POINT -- THERE'S THIS ARGUMENT, RIGHT, ABOUT HOW THE HMSFRS ARE, IN ODD AND PUTATIVELY PAIRWISE-INDEPENDENT WAYS, BIASED TRACERS OF GAS MOTIONS. 
THIS ARGUMENT WAS BROUGHT BOTH IN OFFENSE (THE HMSFRS ARE MOVING IN FUNNY WAYS, SEE, SO YOU CAN'T USE 'EM TO GET CLEAN MEASUREMENTS OF DYNAMICS) AND IN DEFENSE (THE HMSFRS ARE MOVING IN SUCH FUNNY WAYS BECAUSE THEY'RE KINDA COMPACT-ISH OBJECTS WAYS NOW SO MAYBE THEY'RE STARTING TO GET DYNAMICALLY HEATED). 
WHAT WE GET TO SAY TO THIS POINT IS THAT THE HMSFRS, FOR THE MOST PART, SEEM TO BE PRETTY ENTRAINED IN THEIR LOCAL GAS FLOWS, AND BY LOCAL GAS FLOWS WE MEAN THE FLOWS IN A ~0.5KPC PATCH AROUND THEM.
MORE COLLOQUIALLY, MUCH OF THE MOTION OF AN IN INDIVIDUAL HMSFR SEEMS TO BE DUE TO THE GENERAL SLOSHING OF THE ENTIRE TUB, RATHER THAN TO SOME PECULIAR JITTER. (CHECK HOW MUCH IS SAID ABOUT THIS IN THE USUAL SOURCES).

FIRST AND A HALFTH POINT MIGHT BE DANGLING THE NET CHANGE OF THE DISTANCE TO A FEW OBJECTS IN A (E.G., GOODMAN MOLECULAR CLOUD) CATALOG THAT HAVE GOT A RELATIVELY INDEPENDENT DISTANCE AS WELL? JUST TO BRING HOME HOW ONE HAS GOT TO TRY TO DO SOMETHING ABOUT THE HIGH-MAGNITUDE AND LARGE-SCALE STREAMING MOTIONS IF ONE IS TRYING TO DO KINEMATIC DISTANCES. 

SECOND POINT -- DOES THE WAVELENGTH/SHAPE OF THE GENERAL TUB-SLOSH TELL US ANYTHING, RIGHT OFF THE BAT? (CHECK THE TW+ PAPER ON FOURIER MODE DECOMPOSITION OF EXTERNAL GALAXIES TO GET GUIDANCE)
CAN WE READ OFF A NUMBER OF SPIRAL ARMS? IS THERE AN OBVIOUS PATTERN DIFFERENCE BETWEEN FLOCCULENT AND SDW VELOCITY PATTERNS? (LOOK AT THE FOUR CONVERGENCE/DIVERGENCE FIELDS IN DOBBS+)
THE OLD CHI-BY-EYE CAN MAYBE TELL US SOMETHING ABOUT THE FLOCCULENCE VS SDW-NESS OF OUR GALAXY.

THIRD POINT -- CAN WE PUT THESE MOTIONS IN AN EXTRAGALACTIC CONTEXT? E.G. HOW DO THEY MATCH UP WITH THE LARGE DISPERSIONS SEEN IN HALPHA BY RAYMOND AND CO? (ASK RAYMOND ABOUT THIS).


SUMMING UP IN A FIFTH PARAGRAPHS AFTER AN INTRODUCTION AND THREE BODY PARAGRAPHS (UGH, HOW SCHOLASTIC), OUR HIGH-MAGNITUDE STREAMING MOTIONS ARE SPATIALLY PRETTY LARGE, PROBABLY REAL, AND COINCIDENT WITH THE HMSMS OF THE HMSFRS.  
THIS IMPLIES THAT THE HMSFRS ARE MORE TRACING BIGGISH MOTIONS THAN GETTING DYNAMICALLY HEATED. 
WHILE WE CAN SAY VERY LITTLE ABOUT MOTIONS ON SMALL SCALES AND OUR UNDERSTANDING OF OUR FOURIER SPACE COVERAGE, OUR WINDOW FUNCTION, ETC. IS OBVIOUSLY BAD, JUST LOOKING AT A COUPLE SETS OF SIMS AND OUR MAPS SIDE-BY-SIDE SEEMS TO SUGGEST, ~EXTREMELY LOOSELY~, THAT THE GALAXY IS MORE (FLOCCULENT/SDW-ISH) THAN (SDW-ISH/FLOCCULENT). 
IN AN EXTRAGALACTIC CONTEXT, ONE COULD IMAGINE HII REGIONS ENTRAINED IN THESE LARGE FLOWS + RESOLUTION & BEAMSMEARING EFFECTS TEAMING UP TO EXPLAIN SOME OF THOSE LARGE HALPHA DISPERSIONS.
END THE NEW MATERIAL PART OF THE WORK, TRIUMPHANTLY, WITH THESE WILD SPECULATIONS.
