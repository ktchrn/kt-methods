\section{Discussion}
\label{sec:discussion}

\subsection{Unmet assumptions and their potential consequences}
\label{sec:discussion-systematics}
In section HMSFRS, we showed that, empirically, the map of $\vlos$ in PPD space is correct, at least at 93 points coincident with HMSFRs. 
If we add in a few assumptions about the continuity and smoothness of the ISM velocity field, then the map of $\vlos$ in PPD space should also be approximately correct \emph{near} those points. 
Here, the exact value of the spatial scale corresponding to this \emph{nearness} depends on how continuous and smooth the velocity field actually is.
So long as this scale is not too short, we can conjecture that the residual velocity fields shown in figure PIES are also mostly correct, as the HMSFRs are distributed quite densely in two dimensions. 
This is roughly where this ``correctness-by-association'' approach reaches its limits -- most of the full, three-dimensional $\vlos$ map is devoid of HMSFRs.
In any case, this combination of a partial empirical confirmation with a loosely justified heuristic is not a replacement for an analysis and/or intuitive understanding of when, and ideally why, kinetic tomography works and fails.

We unfortunately do not have this sort of analysis. 
To conduct a formal (i.e. analytic) analysis, we would need a parametric description of the ISM, both as a distribution in PPDV space and as the corresponding set of reddening and gas line emission observations; to conduct a more empirical analysis, we would need to several galaxy-scale ISM simulations and the corresponding synthetic observations. 
To the best our knowledge, there are no such parametric descriptions of the ISM.
Performing galaxy simulations and producing synthetic reddening and gas line emission cubes is beyond the scope of this work.


\subsection{Catastrophic outliers}
\label{sec:discussion-catastrophic}
In Section \ref{sec:KT-validation}, we showed that the measured and KT-derived $\vlos$ of 94 of the 99 \Reid HMSFRs agree quite well. 
In this section, we discuss the 5 remaining HMSFRs, which we will refer to as catastrophic outliers.

Figure \ref{fig:outliers}a shows the distribution of standardized residuals of the full 99 HMSFR sample.
This distribution is, for the most part, consistent with a standard normal distribution that has been normalized to integrate to 99, the sample size. 


\subsection{On the plausibility and implications of our gas streaming motions}
While the exact shapes and magnitudes of the streaming motions in our velocity map depend strongly on the assumed galactic parameters (galactocentric distance to the sun, rotation curve, solar motion relative to the rotaion curve), there is no choice of galactocentric distance, \emph{flat} rotation curve, and solar motion that can completely remove them (see Figure 4 pies).
If we limit ourselves to actually plausible galactic parameter sets (panels a, b, c of 4 pies) we see streaming motions on $\sim$ 1 kpc scales with magnitudes in excess of 15 km/sec. 
This is true even if we restrict ourselves to the nearest 5 kpc and is true for any of the three plausible parameter sets. 
These gas motions would be perfectly reasonable if the Milky Way were an obvious grand design spiral such as M83 (cite that nice paper w/the streaming motions). 
Since it is not, we should first ask if they are astrophysically plausible in a a more humble sort of spiral galaxy such as our own. 

Not unexpectedly, we will argue that these spatially extended, high magnitude motions are indeed plausible.
Within 2 kpc of the Sun, the RAVE survey has detected stellar streaming of comparable spatial extent and magnitude (S2012). They interpret these motions as evidence of the gravitaional influence of the Perseus spiral arm.
While the dynamics of stars and gas in s spiral potential will not necessarily be the same, this spiral-induced stellar streming offers a an explanation for gas streaming of an essentially appropriate magnitude.

Streaming motions 3-5 kpc away in the inner galaxy can be associated with the dynamical influence of the Galactic bar. 
This interpretation is bolstered by the similarity of our non-flat rotation curve to that of Clemens (198?), whose shape and deviation magnitude 3-5 kpc from the Galactic center can be explained by the proximity of the bar to the first quadrant tangent points (Incorrect Rotation Curve). 

Our rotation curve is in general also quite similar to many that have been measured by the GHASP survey (CITE). 
The GHASP survey used $\Halpha$ measurements to derive rotation curves for NUMBER disk galaxies with NUMBER kpc of the Milky Way with masses ranging from NUMBER to NUMBER times the mass of the Milky Way.
These rotation curves often show the same sort of $\sim 15$ km/sec, $\sim 1$ kpc velocity ``corrugations'' that we see in our rotation curve. 
The GHASP collaboration derived separate rotation curves for the advancing and receding sides of their galaxies.
Often, the central radii of the velocity corrugations are slightly different on different sides of a galaxy.
These shifts imply that these deviations from flat rotation are caused by non-axisymmetric perturbations, such as spiral arms and/or bars.

Combining these pieces of circumstantial intra- and extragalactic evidence, we can conclude that $\sim 1$ kpc-scale $\sim 15$ km/sec gas streaming motions in a non-grand design spiral galaxy can not immediately be dismissed as unphysical. 
We can also loosely associate these motions with spiral arms and the bar at large (small) and small (large) galactocentric (heliocentric) radii.

If these motions are real, what conclusions can we draw from them without fixing a galactic parameter set? One immediate implication of these streaming motions is that improving the accuracy of kinematic distances will require accounting for large-scale motions beyond the rotation curve. Junichi+ (20??), e.g., estimated the distance distortion induced by assuming a flat rotation curve in a model galaxy non-circular HMSFRs velocities comparable to those of the Reid+09 HMSFRs. 
THERE ARE IMPLICATIONS

The agreement between the gas and expanded HMFSR sample's line-of-sight velocities (see Section US WINNING) suggests that the non-circularity that the non-circularity of the HMSFR's velocities is real. 
An argument against that possibility appeared shortly after the publication of Reid+ (2009), an early analysis of an a subset of the full R+14 sample, in McMillan and Binney (2010); this argument has been repeated with some frequency in discussions of discrepancies between Galactic parameters derived from the HMSFRs and other tracers (the usual suspect and hopefully someone else too).
If the HMSFRs' motions are real and due to orbital ellipticity, the argument goes, then the implied ellipticity of their orbits is higer than that of stars that are young but already out of their birth clouds. 
Since HMSFRs are temoraly closer to collisional, and hence dynamically colder, dense gas than the collisionless, and hence dynamically hotter, young stars, this dynamical state of things is not physically plausible. 
McMillan and Binney (2010) suggest some alternatives, including a bias towards bluer line-of-sight velocities, to reduce the required orbital ellipticity and bring the velocity dispersions of the two ostensibly similar populations closer together.

The agreement between the expanded HMSFR sample's line-of-sight velocities and our (completely independently) derived map suggests that the HMSFRs deviations from flat rotation are real and not due to measurement biases.
The spatial scale and similarity to spiral- and bar-induced streaming of non-circular motions in our map also suggests a resolution to the velocity dispersion inconsistency. The HMSFRs are moving coherently, with the gas they are embedded in, rather than randomly. Once the gas and theembedded HMSFRs leave the perturbations they are currently passing through, their apparent velocity disperion as a population will presumably fall to something closer to that of the current generation of young stars.

Conversely, the McMillan and Binney argument and the agreement between the HMSFR velocities and ourmap combine into yet another piece of circumstancial evidence in favor of our streaming motions being associated with spiral- and bar-induced perturbations. 
If interpreting the HMSFR motions as stable elliptical orbits is out of the qustion, then interpreting the (if anything dynamically even colder) gas motions as stable elliptical motions must be even less acceptable.
