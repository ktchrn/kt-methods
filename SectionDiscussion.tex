\section{Discussion}
\label{sec:discussion}

\subsection{Ways in which our assumptions can fail}
\label{sec:discussion-systematics}
WHAT ARE THE ASSUMPTIONS?
ONE OBVIOUS ASSUMPTION COMBO -- THE GAS AND DUST ARE, RESPECTIVELY, GOOD PICTURES OF THE PPV AND PPP ISM. THIS GETS BROKEN IN LOTS OF INTERESTING WAYS

ANOTHER -- THE VELOCITY GRADIENT THROUGHOUT A SINGLE DISTANCE "BIN" IN A "SKEWER" IS NEGLIGIBLE; THE DISTRIBUTION OF VELOCITIES THROUGHOUT A BIN CAN BE ADEQUATELY DESCRIBED AS A SINGLE SOLITARY GAUSSIAN. I DON'T HAVE AN OBVIOUS EXAMPLE OF THIS HAPPENING, BUT POINTING OUT THAT DISTANT PARTS OF THE PPP CUBE GET ASSIGNED SUPER WIDE GAUSSIANS IS A THING I COULD DO; STRIPES IN THE "ROTATION CURVE IS RIGHT" LV DIAGRAM ARE ALSO KINDA EVIDENCE OF THIS.

NEIGHBORING PIXELS SHOULD HAVE SIMILAR VELOCITIES. THERE ARE LOGICAL REASONS TO EXPECT THIS TO NOT BE TRUE IN A WAY THAT GENERALLY DEPENDS ON DISTANCE / CONSIDER THE POTENTIAL EXISTENCE OF SPIRAL SHOCKS, BUT NO HARD EVIDENCE OF IT DOING A BAD THING. A SEPARATE AND POSSIBLY TRICKY POINT IS THAT WE MOTIVATE HAVING SPRINGS BY SAYING THAT WE WANT TO HAVE A UNIQUE SOLUTION. WE (1) DON'T SHOW THAT THE REGULARIZED SOLUTION IS UNIQUE, OR PUT ANOTHER AWAY THAT THE REGULARIZATION IS ACTUALLY STRONG ENOUGH TO ENSURE A UNIQUE SOLUTION AND (2) WE DON'T HAVE EVEN A HEURISTIC ARGUMENT FOR WHY A SOLUTION WHICH IS REGULARIZED HARD ENOUGH TO BE UNIQUE SHOULD BE ONE THAT WE TRUST AT ALL.


\subsubsection{Ways in which the PPP and PPV cubes can fail to be accurate depictions of the PPP and PPV ISM}
EACH ONE CAN BE WRONG, ON ITS OWN

THE PPP AND PPV CUBES CAN BE INDIVIDUALLY ``RIGHT'' OVER SOME PIECES OF THE ISM BUT WRONG OVER-ALL 
Figures \ref{fig:tracer_syst_values} and \ref{fig:tracer_syst_spatial}

WHAT DOES THIS DO TO THE SOLUTION? NOT MUCH APPARENTLY, SINCE THE SPOTS WHERE WE KNOW WE'RE RIGHT (HMSFRs) ARE EXACTLY WHERE TOTAL COLUMN COMPARISONS SAY WE SHOULD BE VERY WRONG. ALSO THE SOLUTION IS LIMITED TO ABS(B) < 10, SO WE CUT OUT MOST OF THE BLUE STUFF. I GUESS THIS ASSUMPTION IS LESS IMPORTANT THAN I HAD ASSUMED? 

\subsubsection{How wrong is it to assume that the velocity distribution of the ISM in a single PPP voxel is pretty close to a single relatively narrow Gaussian?}
IT'S MOSTLY OKAY, BUT BREAKS DOWN FOR FAR AWAY PIXELS. LUCKILY (?) THESE ARE MOSTLY OUTSIDE OF THE RANGE IN WHICH THE PPP MAP IS GOOD ANYWAY

BASICALLY DOESN'T MATTER TOO MUCH

\subsubsection{Possible problems with the regularization scheme}
TWO MAIN CASES -- REGULARIZATION IS TOO WEAK AND WE DON'T ACTUALLY HAVE A UNIQUE SOLUTION, WHAT THEN; REGULARIZATION IS TOO STRONG AND WE DISTORT STUFF, WHAT THEN?

HANDWAVE AND HEDGE ABOUT THE FIRST ONE BECAUSE WE NEVER CLAIM TO HAVE AN ACTUALLY UNIQUE SOLUTION

WE DON'T HAVE GOOD TESTS THAT COULD TELL US IF WE'RE MAKING THE DIFFUSE ISM MOVE IN UNNECESSARILY FUNKY WAYS. THE ROTATION CURVE, WHICH IS DERIVED MOSTLY FROM THE BULK GENERAL STUFF, SEEMS PRETTY SIMILAR IN BOTH CASES SO IT CAN'T BE TOO BAD? 

THIS IS ONE OF THE REASONS WHY WE DON'T WANT TO RELEASE THESE MAPS

\subsection{Systematics (THIS SECTION IS DEPRECATED)}
THIS SECTION IS INCOHERENT AND NEEDS BETTER QUANTIFICATION

PIECES OF IT MAY END UP IN BETTER-PLANNED SECTIONS UP ABOVE

ALSO THESE ARE "INCORRECT ASSUMPTIONS", NOT "SYSTEMATICS"

In section \ref{sec:KT-method}, we make assumptions about the ISM and our observations of it that are not expected to apply over the entire galaxy. Here, we discuss the expected effect of these assumptions on our solution.
\subsubsection{Individual sources}
DARK GAS (missing CO and \atomH self-absorption)
One of the core assumptions of our method is that the intensity of \atomH and CO emission at an on-sky position and line-of-sight velocity scales linearly with the amount of ISM occupying that on-sky position and line-of-sight velocity. This assumption is incorrect when the optical depth of either emitting species is greater than unity and when not all ISM mass is traced by the combination of \atomH and CO emission. 

We expect the effects of optical depth effects on our estimate of the distribution of ISM mass over PPV space to be negligible over most of the PPV cube. This is despite the facts that the CO (J=1-0) transition is practically always optically thick and that there are is clear evidence of \atomH self-absorption in the \atomH emission cube. When converting optically thick CO emission to a \molH mass, it is more correct to scale the integral of the emission over the CO line profile (CITE?). However, since the individual CO line profiles in our data cube have width of order a resolution element, simply scaling the CO emission cube is approximately equivalent to applying the more correct prescription and assigning the result to each line profile's central velocity. We neglect \atomH self-absorption because most of the ISM mass in the environment towards which the required \atomH column is usually present, dense molecular clouds, is traced by CO emission. 

The second case, in which not all (non-ionized) ISM mass is traced by the combination of \atomH and CO emission, is more clearly a problem. It is widely known that the part of a molecular cloud in which \molH is spatially coincident with well-shielded CO is surrounded by a layer of CO-dark gas. In this layer, most hydrogen is in the form of \molH, and hence not traced by \atomH emission, but there is not enough shielding to prevent the photodissociation of CO. The exact extent of and mass contained in this layer is still debated (CITE AND CITE; THEY DISAGREE!), but its existence is widely accepted (CITE BOLATTO REVIEW). The amount of CO-dark molecular gas associated with a molecular cloud in the vicinity of the Sun is thought to be about $1/10$ to $1/2$ of the total mass of the cloud. The spatial variation of the CO-dark mass fraction across a molecular cloud is considerably less well-constrained.

%NEEDS DATA
 We can see evidence for CO-dark gas in our maps when we compare the position-position maps obtained by integrating the PPP and PPV cubes over distance and line-of-sight velocity, respectively, towards relatively unconfused molecular clouds. In figure REF:CO-DARK-PERSEUS, we show the spatially-resolved relative difference between the PPP- and PPV-derived integrated maps towards the Perseus molecular cloud. DESCRIBE THE VARIATIONS. Since the magnitude of this effect is expected to vary from molecular cloud to molecular cloud FOR REASONS (CITE), we note its presence but do not attempt to correct for it. We believe this to be one of the major systematic uncertainties affecting our map of the ISM in PPPV space. The magnitude of the effect of CO-dark gas on the PPP cube of velocities is expected to be lower, though not negligible. QUANTIFY?

The other core assumption of our method is that the PPP reddening cube of \citet{Green_2015}  is an accurate representation of the spatial structure of the Milky Way ISM. For observational and astrophysical reasons, this is non-trivially not the case. Most obviously, the GSF+2015 reddening cube is limited in its distance extent. The maximum reliable distance in the PPP cube varies between 3 and 12 kpc (see Figure 8 of GSF+2015), while the PPV cube we are trying to reshuffle it into is an integration to arbitrarily large distances. This difference between the PPP and PPV cubes means that we could potentially be associating parts of the PPP cube with parts of the PPV cube that are actually farther away.

The PPP cube is also not always accurate. Computing the distribution of reddening with distance in a single on-sky pixel of the PPP cube involves simultaneously solving for the distance, reddening, and intrinsic appearence of the stars in that on-sky pixel. For a single star, these parameters are clearly completely degenerate. CITET GREEN+2014, among others (maybe CITE the usual suspects on this), have shown that this degeneracy can be broken by jointly analyzing all of the stars along a given line of sight, given certain conditions on the individual stars' signal-to-noise ratios and how well the sightline is sampled. The effectiveness of this degeneracy breaking is, in large part, what determines the accuracy of the distance-reddening profile. Parts of the PPP cube with relatively low signal-to-noise ratio stellar observations (i.e. highly reddened areas) or relatively low stellar densities will have less accurate distance-reddening profiles. These inaccuracies translate directly into distance-velocity profile inaccuracies in our solutions. 

As is described in section \ref{sec:data}, we convert this reddening cube into a spatial map of the ISM by scaling it by a constant factor. This constant factor is the product of the extinction to color excess ratio $R_V$ and the dust-to-gas ratio (DGR). By assuming this factor is constant, we ignore the fact that $R_V$ and the DGR are both known to vary. We expect $R_V$ variation to be mostly inconsequential. CITET SCHLAFLY 2016 have shown that $R_V$ towards the vast majority of typical Milky Way stars is between 3 and 3.7. This variation in $R_V$ translates to a 20\% systematic uncertainty on $A_V$.

DGR variations have the potential to introduce larger spatially-varying biases into our ISM mass PPP cube, but these biases are still expected to be smaller than those induced by CO-dark gas. Studies of the Milky Way DGR using dust and gas emission (e.g. CITE PLANCK+Fermi, REACH+2015) and elemental depletions (CITE JENKINS) find that the DGR increases by a factor of about 3 from diffuse neutral to dense molecular gas. 

REGULARIZATION

\subsubsection{Cumulative effect}
REGIMES WE DEFINITELY BELIEVE

REGIMES WE DON'T QUITE BELIEVE YET

\subsection{Catastrophic outliers}
\label{sec:discussion-catastrophic}

\subsection{Something else to talk about}