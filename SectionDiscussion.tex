\section{Discussion}
\label{sec:discussion}

\subsection{Catastrophic outliers}
\label{sec:discussion-catastrophic}
In Section \ref{sec:KT-validation}, we showed that the measured and KT-derived $\vlos$ values  of 94 of the 99 \Reid HMSFRs agree with each other. 
In this section, we discuss the five HMSFRs where they do not agree.

Panel (a) of Figure \ref{fig:outliers} shows the distribution of standardized residuals of the full 99 HMSFR sample along with a standard normal distribution that has been scaled to integrate to the number of HMSFRs.
There are five HMSFRs, which have been indicated on the figure in orange, that are clearly separated from the rest. 
These are the HMSFRs for which we consider the measured and KT-derived $\vlos$ values to disagree.
We will refer to these five as the outliers.

We have not been able to find an obvious reason for why these five HMSFRs, specifically, are outliers. 
None of their properties are extreme -- they are not all clumped together; they are not particularly close to or far from the Sun; the total dust and gas columns along their sightlines are typical for HMSFRs in their vicinity; they are reasonably close to their nearest neighbors, none of which are outliers; the differences between their measured $\vlos$ values and those of their close neighbors are reasonable, as are the differences between the magnitudes of their velocity three-vectors; they do not all sit in the same spiral arm (\Reid). 
There is no clear distinction between an outlier and its non-outlier neighbors. 

While there does not appear to be a way to predict whether a specific HMSFR will definitely be an outlier, we have found that HMSFRs with $0^\circ < \glon < 35^\circ$ and Galactocentric radii ($\RGC$) between 3 and 5 kpc are more likely to be outliers. 
The distributions of outlier and non-outlier $\glon$ and $\RGC$ are shown in panels (b) and (c) of Figure \ref{fig:outliers}, and the locations of the outlier HMSFRs and some nearby non-outliers are shown in panel (d).
By visual inspection, the outliers' $\glon$ and $\RGC$ distributions clearly deviate from those of the full sample.

We can quantify this deviation by assuming that any HMSFR is equally likely to be one of the five outliers and computing the probability of finding all five in a randomly chosen subsample of the same sizes as one of our two selections.
Since a single HMSFR can not be present in a sample twice, the appropriate distribution to use for this calculation is the hypergeometric distribution, which assumes that the subsample is chosen without replacement.
The probabilities of finding all five outliers in a random subsample the size of the $\glon$ and $\RGC$ selections are both well below 1\%.

Since every HMSFR in the $\RGC$ selection is also in the $\glon$ selection, this calculation does not tell us which selection is responsible for the enhance outlier probability. 
To try to differentiate between the two, we can repeat the above exercise with the $\RGC$ selection as the subsample and the $\glon$ selection, rather than the full set of 99 HMSFRs, as the population. 
The probability of finding all five outliers in a random 17-element subsample of the 31-element $\glon$ selection is about 4\%. 
This probability is low, certainly low enough to suggest that there may be an enhancement in the outlier rate in the $\RGC$ selection over the $\glon$ selection's rate, but not enough for that to be a strong conclusion.
This is especially true considering that we have entirely neglected the distance and outlier/non-outlier classification uncertainties.

We are trying to distinguish between the two selections because they suggest different causal explanations for the increased likelihood of being an outlier. 
If being in the $\glon$ selection is responsible for most of the effect, the likely cause is increased line-of-sight confusion. 
The closer to $\glon$ of $0^\circ$ a sightline passes, the more matter it typically intersects near the Galactic center and, assuming the Galactic disk is roughly circular, the more matter it passes through on the opposite side of the Galaxy. 
If instead being in the $\RGC$ selection is what matters, the cause is likely to be the complexity of ISM dynamics near the Galactic center.
The distance resolution of the PPD cube declines with distance, so it would not necessarily be surprising if we were unable to spatially resolve flows with small spatial scales.
The end of the Galactic bar, coincidentally, is generally considered to fall right in the middle of the five outliers \cite{2016arXiv160207702B}.

If we treat sightlines with HMFSRs as typical, the implied catastrophic failure rate in the inner galaxy is somewhere in the vicinity of 30\%. 
Conversely, there are no outliers outside of the inner galaxy, regardless of whether that is defined as an $\glon$ range or as a $\RGC$ range.

\subsection{Unmet assumptions and their potential consequences}
\label{sec:discussion-systematics}
The procedure described in Section Method is based on four main assumptions about the relation between the two datasets and the shape of the ISM in PPDV space. 
The first assumption is that the distribution of the ISM, in absolute units (e.g. number of protons), in PPD and PPV space is a linear scaling of the input reddening and gas line emission cubes.
The second assumption is that PPD and PPV cubes are projections of the same region of PPDV space, or equivalently that all the ISM contained in one is also contained in the other.
The third assumption is that the PPV cube can be reproduced by reprojecting the PPD cube -- ``lifting'' to PPDV space and projecting the result back down -- to PPV space.
This lift is assumed to have a simple description, a one-to-one function convolved with a Gaussian kernel in velocity space.
The fourth assumption is that voxels in PPD space that share a $\glon$ or $\glat$ boundary have similar central line-of-sight velocities.

These assumptions fail to hold, to different degrees, in the actual case we are applying the procedure to. 
The reddening-to-matter and $\atomHI$- and CO-to-matter conversion is not a single linear relationship. 
The PPV cube includes matter on the far side of the Galaxy while the PPD cube stops at or before the level of the Galactic center.
The PPV cube can not be a reprojection of the PPD cube if the two are not tracing the same matter, and furthermore there will be voxels in the PPD cube where the lift to PPDV space is not the simple one we have assumed.
Finally, since the ISM can contain shocks, the $\vlos$ field can contain discontinuities.

Despite these unmet assumptions, the solution the procedure produces is remarkably accurate (see Section \ref{sec:KT-validation}).
Evidently, the procedure is robust to some level of assumption-breaking. 
The clustering of poorly-matched HMSFRs in the inner Galaxy, which is discussed in Section \ref{sec:discussion-catastrophic} suggests that there are parts of the sky where the assumptions fail strongly enough that the solution becomes noticeably less accurate. 
As is also discussed in that section, it is unclear whether the more important effect is being at low $\glon$, where the pathlength difference between the PPD and PPV cubes is highest, or being near the tip of the Galactic bar, where the velocity field is expected to be particularly complex.
In either case, sightlines in the inner Galaxy are more likely to not meet the assumptions behind KT than sightlines elsewhere. 
The $\vlos$ field we derive for the inner 5 kpc or so of the Galaxy should be treated with caution.

What we unfortunately do not know is whether there are other regimes in which our assumptions are violated strongly enough to produce systematic errors. 
For example, if the more important of the two effects in the inner Galaxy is the mismatch between the PPD and PPV cubes, then we should also be more skeptical about the depths of molecular clouds, which can ``disappear'' from the PPD cube due to a drop in the map's completeness at high reddening.
The opposite regime, diffuse atomic gas, is completely untested by the HMSFRs.
Since diffuse gas occupies a higher fraction of the ISM by volume than dense gas, it would 


POINTS:
    - works pretty well
    - fails in a specific spot
    - the nature of that specific spot suggests some failure modes, and here they are
    - our understanding of the point we just made is imperfect, and in particular we don't have a quantitative understanding of how solutions break down as these failed assumptions start piling up
    - if we wanted to be quantitative about it and get uncertainties, biases, probabilities of failure (which we would need to have to be comfortable with publicly releasing the solution), we would do artificial data tests
    - these artificial data tests would involve some pretty complicated steps, so we're not going to do them right now
    - that does mean that we aren't really doing anything with regions that we haven't been able to check yet in this paper, will need to come up with new ways of checking them



\subsection{On the astrophysical plausibility and implications of our gas streaming motions}
\label{sec:discussion-plausibility}
While the exact shapes and magnitudes of the streaming motions in our velocity map depend on the assumed galactic parameters (Galactocentric distance to the sun, rotation curve, solar motion relative to the rotation curve), there is no choice of Galactocentric distance, rotation curve, and solar motion that can completely remove them (see Section \ref{sec:rotation_discussion} and Figure \ref{fig:six_pies}).
If we limit ourselves to rotation curves that are flat outside of the inner galaxy (panels a, b, e, f of Figure \ref{fig:six_pies}), we see streaming motions on $\sim$ 1 kpc scales with magnitudes in excess of 15 km/sec. 
These gas motions would be perfectly reasonable if the Milky Way were an obvious grand design spiral such as M51 \citep{Meidt_2013}. 
Since it is not, we should first ask if they are astrophysically plausible in a more humble sort of spiral galaxy such as our own. 

We will argue that these spatially extended, high magnitude motions are indeed plausible.
Within 2 kpc of the Sun, the RAVE survey has detected stellar streaming of comparable spatial extent and magnitude (S2012). They interpret these motions as evidence of the gravitational influence of the Perseus spiral arm.
While the dynamics of stars and gas in s spiral potential will not necessarily be the same, this spiral-induced stellar streaming offers a an explanation for gas streaming of an essentially appropriate magnitude.

Streaming motions 3-5 kpc away in the inner galaxy can be associated with the dynamical influence of the Galactic bar. 
This interpretation is bolstered by the similarity of our non-flat rotation curve to that of Clemens (198?), whose shape and deviation magnitude 3-5 kpc from the Galactic center can be explained by the proximity of the bar to the first quadrant tangent points (Incorrect Rotation Curve). 

Our rotation curve is in general also quite similar to many that have been measured by the GHASP survey (CITE). 
The GHASP survey used $\Halpha$ measurements to derive rotation curves for NUMBER disk galaxies with NUMBER kpc of the Milky Way with masses ranging from NUMBER to NUMBER times the mass of the Milky Way.
These rotation curves often show the same sort of $\sim 15$ km/sec, $\sim 1$ kpc velocity ``corrugations'' that we see in our rotation curve. 
The GHASP collaboration derived separate rotation curves for the advancing and receding sides of their galaxies.
Often, the central radii of the velocity corrugations are slightly different on different sides of a galaxy.
These shifts imply that these deviations from flat rotation are caused by non-axisymmetric perturbations, such as spiral arms and/or bars.

Combining these pieces of circumstantial intra- and extragalactic evidence, we can conclude that $\sim 1$ kpc-scale $\sim 15$ km/sec gas streaming motions in a non-grand design spiral galaxy can not immediately be dismissed as unphysical. 
We can also loosely associate these motions with spiral arms and the bar at large (small) and small (large) Galactocentric (heliocentric) radii.

If these motions are real, what conclusions can we draw from them without fixing a galactic parameter set? One immediate implication of these streaming motions is that improving the accuracy of kinematic distances will require accounting for large-scale motions beyond the rotation curve. Junichi+ (20??), e.g., estimated the distance distortion induced by assuming a flat rotation curve in a model galaxy non-circular HMSFRs velocities comparable to those of the Reid+09 HMSFRs. 
THERE ARE IMPLICATIONS

The agreement between the gas and expanded HMFSR sample's line-of-sight velocities (see Section US WINNING) suggests that the non-circularity that the non-circularity of the HMSFR's velocities is real. 
An argument against that possibility appeared shortly after the publication of Reid+ (2009), an early analysis of an a subset of the full \Reid sample, in McMillan and Binney (2010); this argument has been repeated with some frequency in discussions of discrepancies between Galactic parameters derived from the HMSFRs and other tracers (the usual suspect and hopefully someone else too).
If the HMSFRs' motions are real and due to orbital ellipticity, the argument goes, then the implied ellipticity of their orbits is higher than that of stars that are young but already out of their birth clouds. 
Since HMSFRs are temporaly closer to collisional, and hence dynamically colder, dense gas than the collisionless, and hence dynamically hotter, young stars, this dynamical state of things is not physically plausible. 
McMillan and Binney (2010) suggest some alternatives, including a bias towards bluer line-of-sight velocities, to reduce the required orbital ellipticity and bring the velocity dispersions of the two ostensibly similar populations closer together.

The agreement between the expanded HMSFR sample's line-of-sight velocities and our (completely independently) derived map suggests that the HMSFRs deviations from flat rotation are real and not due to measurement biases.
The spatial scale and similarity to spiral- and bar-induced streaming of non-circular motions in our map also suggests a resolution to the velocity dispersion inconsistency. The HMSFRs are moving coherently, with the gas they are embedded in, rather than randomly. Once the gas and the embedded HMSFRs leave the perturbations they are currently passing through, their apparent velocity dispersion as a population will presumably fall to something closer to that of the current generation of young stars.

Conversely, the McMillan and Binney argument and the agreement between the HMSFR velocities and our map combine into yet another piece of circumstantial evidence in favor of our streaming motions being associated with spiral- and bar-induced perturbations. 
If interpreting the HMSFR motions as stable elliptical orbits is out of the question, then interpreting the (if anything dynamically even colder) gas motions as stable elliptical motions must be even less acceptable.
