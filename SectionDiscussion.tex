\section{Discussion}
\label{sec:discussion}

\subsection{Unmet assumptions and their potential consequences}
\label{sec:discussion-systematics}
The mathematical model behind kinetic tomography contains some simplifying assumptions about the input data and the structure of the ISM in PPDV space. 
We assume the input data, meaning the reddening and gas line emission cubes, are tracing the same bits of ISM and can be converted to distributions of matter in PPD and PPV space, respectively, by the scalings and additions described in Section DATA.
We assume that the distribution of VLOS in a voxel of the PPD cube is Gaussian and that the centers of ($\glons$, $\glat$)-neighboring velocity Gaussians are close in the square-difference sense. 
Ideally, we would have a way of formally, meaning more-or-less analytically, characterizing how the kinetic tomography solution degrades as each of these assumptions becomes less valid.
This would serve as a more general complement to our empirical tests of the kinetic tomography solution at the locations of HMSFRs (see section HMSFRs).

Since we are interested in possible systematics in a PPDV map of the ISM, we are only interested in those failure modes of our assumptions that can occur in the actual ISM; an analytic characterization of complex systematics in an analysis of the ISM of an entire galaxy requires an analytic description of the ISM of an entire galaxy.
Unfortunately, these sorts of analytic descriptions do not currently exist.
An alternative, semi-empirical, way to estimate the accuracy of kinetic tomography over a broad range of conditions is to apply it to synthetic observations of simulated galaxies and then explore what factors the accuracy of the results depends on. 
Most of the steps involved in that procedure are, on their own, beyond the scope of this work. 
This means that we do not have an estimate of the systematic or random uncertainties of the PPDV solution or the map of VLOS as a function of $\glon$, $\glat$, and distance. 
Instead, we will merely list some of the ways in which our assumptions are either known or are likely to fail and advise the reader to be careful in their interpretation of the VLOS maps in figure PIES, particularly away from the points where we have been able to empirically validate it.

First, we discuss failures of the assumptions about the data -- that the conversions from reddening and gas line emission to distributions of ISM mass in PPD and PPV space given in section DATA are accurate and that every bit of mass that is present in either one of the PPD and PPV cubes is also present in the other. 
The most egregious failure mode is that the PPD cube covers a limited distance range while the PPV cube extends to arbitrarily large distance. 
The distance range of the PPD cube is set by the three-dimensional distribution of stars that have been detected by PanSTARRS, the survey used to build the reddening map.
The distance to the farthest star along a ($\glon$, $\glat$) sightline depends on both the intrinsic distribution of stars and the amount of dust along the line of sight. 
As a result, the PPD cube does not include interstellar matter on the far side of the Galaxy (CITE GREEN), even though this matter is present in the PPV cube.

The remaining failure modes are less egregious than missing more than half of the galaxy, but may still cause some problems.
They are: the presence of dark gas, which can be some combination of optically thick $\atomH$ and CO-dark $\molH$ (cite, e.g. Fermi+Planck); variation of the conversion factor between CO emission and amount of $\molH$ (cite Bolatto review?); and variation of the dust-to-gas ratio (many people one could cite). 
Each of these could cause the mass of a bit of ISM in the PPD and PPV cubes to differ by a factor of around two.




\subsection{Catastrophic outliers}
\label{sec:discussion-catastrophic}
FOR WHATEVER REASON, THE BAD MASERS JUST REALLY DON'T LOOK VERY DIFFERENT FROM THE GOOD MASERS. 

THE ONE THING, WHICH IS HARD TO QUANTIFY BUT MAAAAAYBE SORT OF THERE IS THAT THE OUTLIERS' VELOCITIES ARE PRETTY DIFFERENT FROM THE VELOCITIES OF THEIR NEIGHBORS? 

HOW CAN WE TRY TO QUANTIFY THIS -- PICK A NUMBER OF NEAREST NEIGHBORS (SELECTED IN 3D), LOOK AT DISTRIBUTION OF STANDARD DEV OF THOSE NEIGHBORS' VELOCITIES, SEE WHERE THE OUTLIERS FALL IN IT

IF, AND THAT'S STILL AN IF, "THESE ARE DIFFERENT FROM THEIR NEIGHBORS AND THEIR NEIGHBORS LOOK LIKE THE GAS" ENDS UP BEING TRUE, HOW CAN WE INTERPRET THAT? WELL, MAYBE THE OUTLIERS ARE THE ONES WHAT GOT KICKED IN THE PROCESS OF FORMATION, SO THEIR KINEMATICS HAVE DECOUPLED FROM THE GAS.

AND ANOTHER THING -- ARE THESE SOURCES SYSTEMATICALLY _OLDER_? BECAUSE YOU'D KIND OF EXPECT THE OLD STUFF TO HAVE DECOUPLED MORE FROM THE GAS, RIGHT? MAYBE THERE'S A DIFFERENCE IN TERMS OF MASER TYPE, AS WELL?

\subsection{On the plausibility and implications of our gas streaming motions}
While the exact shapes and magnitudes of the streaming motions in our velocity map depend strongly on the assumed galactic parameters (galactocentric distance to the sun, rotation curve, solar motion relative to the rotaion curve), there is no choice of galactocentric distance, \emph{flat} rotation curve, and solar motion that can completely remove them (see Figure 4 pies).
If we limit ourselves to actually plausible galactic parameter sets (panels a, b, c of 4 pies) we see streaming motions on $\sim$ 1 kpc scales with magnitudes in excess of 15 km/sec. 
This is true even if we restrict ourselves to the nearest 5 kpc and is true for any of the three plausible parameter sets. 
These gas motions would be perfectly reasonable if the Milky Way were an obvious grand design spiral such as M83 (cite that nice paper w/the streaming motions). 
Since it is not, we should first ask if they are astrophysically plausible in a a more humble sort of spiral galaxy such as our own. 

Not unexpectedly, we will argue that these spatially extended, high magnitude motions are indeed plausible.
Within 2 kpc of the Sun, the RAVE survey has detected stellar streaming of comparable spatial extent and magnitude (S2012). They interpret these motions as evidence of the gravitaional influence of the Perseus spiral arm.
While the dynamics of stars and gas in s spiral potential will not necessarily be the same, this spiral-induced stellar streming offers a an explanation for gas streaming of an essentially appropriate magnitude.

Streaming motions 3-5 kpc away in the inner galaxy can be associated with the dynamical influence of the Galactic bar. 
This interpretation is bolstered by the similarity of our non-flat rotation curve to that of Clemens (198?), whose shape and deviation magnitude 3-5 kpc from the Galactic center can be explained by the proximity of the bar to the first quadrant tangent points (Incorrect Rotation Curve). 

Our rotation curve is in general also quite similar to many that have been measured by the GHASP survey (CITE). 
The GHASP survey used $\Halpha$ measurements to derive rotation curves for NUMBER disk galaxies with NUMBER kpc of the Milky Way with masses ranging from NUMBER to NUMBER times the mass of the Milky Way.
These rotation curves often show the same sort of $\sim 15$ km/sec, $\sim 1$ kpc velocity ``corrugations'' that we see in our rotation curve. 
The GHASP collaboration derived separate rotation curves for the advancing and receding sides of their galaxies.
Often, the central radii of the velocity corrugations are slightly different on different sides of a galaxy.
These shifts imply that these deviations from flat rotation are caused by non-axisymmetric perturbations, such as spiral arms and/or bars.

Combining these pieces of circumstantial intra- and extragalactic evidence, we can conclude that $\sim 1$ kpc-scale $\sim 15$ km/sec gas streaming motions in a non-grand design spiral galaxy can not immediately be dismissed as unphysical. 
We can also loosely associate these motions with spiral arms and the bar at large (small) and small (large) galactocentric (heliocentric) radii.

If these motions are real, what conclusions can we draw from them without fixing a galactic parameter set? One immediate implication of these streaming motions is that improving the accuracy of kinematic distances will require accounting for large-scale motions beyond the rotation curve. Junichi+ (20??), e.g., estimated the distance distortion induced by assuming a flat rotation curve in a model galaxy non-circular HMSFRs velocities comparable to those of the Reid+09 HMSFRs. 
THERE ARE IMPLICATIONS

The agreement between the gas and expanded HMFSR sample's line-of-sight velocities (see Section US WINNING) suggests that the non-circularity that the non-circularity of the HMSFR's velocities is real. 
An argument against that possibility appeared shortly after the publication of Reid+ (2009), an early analysis of an a subset of the full R+14 sample, in McMillan and Binney (2010); this argument has been repeated with some frequency in discussions of discrepancies between Galactic parameters derived from the HMSFRs and other tracers (the usual suspect and hopefully someone else too).
If the HMSFRs' motions are real and due to orbital ellipticity, the argument goes, then the implied ellipticity of their orbits is higer than that of stars that are young but already out of their birth clouds. 
Since HMSFRs are temoraly closer to collisional, and hence dynamically colder, dense gas than the collisionless, and hence dynamically hotter, young stars, this dynamical state of things is not physically plausible. 
McMillan and Binney (2010) suggest some alternatives, including a bias towards bluer line-of-sight velocities, to reduce the required orbital ellipticity and bring the velocity dispersions of the two ostensibly similar populations closer together.

The agreement between the expanded HMSFR sample's line-of-sight velocities and our (completely independently) derived map suggests that the HMSFRs deviations from flat rotation are real and not due to measurement biases.
The spatial scale and similarity to spiral- and bar-induced streaming of non-circular motions in our map also suggests a resolution to the velocity dispersion inconsistency. The HMSFRs are moving coherently, with the gas they are embedded in, rather than randomly. Once the gas and theembedded HMSFRs leave the perturbations they are currently passing through, their apparent velocity disperion as a population will presumably fall to something closer to that of the current generation of young stars.

Conversely, the McMillan and Binney argument and the agreement between the HMSFR velocities and ourmap combine into yet another piece of circumstancial evidence in favor of our streaming motions being associated with spiral- and bar-induced perturbations. 
If interpreting the HMSFR motions as stable elliptical orbits is out of the qustion, then interpreting the (if anything dynamically even colder) gas motions as stable elliptical motions must be even less acceptable.
