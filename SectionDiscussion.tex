\section{Discussion}
\label{sec:discussion}

\subsection{Ways in which our assumptions can fail}
\label{sec:discussion-systematics}
In our description of kinetic tomography, we make three main assumptions -- we can linearly scale data cubes of gas and dust tracers to get accurate PPV and PPP maps of the ISM, the distribution of velocities associated with a single PPP voxel can be well-approximated by a narrow Gaussian, and insisting that the central velocities of neighboring PPP voxels should be similar regularizes the PPPV solution in a valid and useful way.
We have not made the claim that kinetic tomography will be in any sense \emph{optimal} (CITE DUSTIN'S APRIL FOOLS PAPER) if these assumptions are true, but we have made the claim that kinetic tomography will be \emph{pretty good}. 
To what extent do these assumptions appear to be true of the data we have to work with and how does their failure affect the kinetic tomography solution?

The first assumption, that linearly scaling data cubes of gas and dust tracers yields accurate PPV and PPP maps of the ISM, has several localized failure modes and two global failure modes. 
The local failure modes are due to observational and astrophysical phenomena, such as $\HI$ self-absorption and variations in the dust reddening law, which produce localized non-linearities and variations in the conversion from tracer to mass. 
The global failure modes arise from the maximum distance mapped by the dust tracer cube being shorter than the maximum distance mapped by the gas tracer cubes. 
By construction, the gas tracers cubes map all gas within a certain velocity range. 
Gas within this velocity range can be arbitrarily far away. 
As a result, the gas tracer cubes map out the entire galaxy.
The dust tracer cube is built by analyzing reddening towards stars. If there are no measurements of stars beyond some distance, the dust tracer cube cuts off.
This cutoff can happen due to large reddenings (e.g. on the far side of the Galactic center) or the spatial structure of the Milky Way's stellar population (e.g. at large heights above the Galactic plane).

We can get an idea of the magnitude and extent of these different PPV-PPP cube mistmatches by comparing the projections of the PPV and PPP cubes to the $\glon$-$\glat$ plane. 
The distribution of gas-derived versus reddening-derived total columns is shown in figure \ref{fig:tracer_syst_values}. 
The two sets of total columns are mostly consistent with each other, i.e. lie along the orange line. 
There are two regimes in which the total columns are not consistent -- at gas columns below $10^{21}$ cm$^{-2}$ (marked in  blue) and at gas columns above CHECK THE NUMBER $3 \times 10^{22}$ cm$^{-2}$ (marked in red). 
In both cases, the reddening-derived total column is noticeably smaller than the gas-derived total column.

The spatial distribution of sightlines in the blue and red areas is shown in figure \ref{fig:tracer_syst_spatial} along with the on-sky positions of the HMSFRs from section \ref{sec:KT-validation}. 
The low-column sightlines with less reddening-traced than gas-traced total column, which are marked in blue, all contain gas at large heights above the Galactic plane. 
This gas is not well-traced by reddening because (1) there are fewer stars at large heights above the Galactic plane, limiting the maximum distance accurately described by the PPP cube at high latitude; and (2) the dust-to-gas ratio at large heights above the Galactic plane is thought to be lower than it is in the Galactic plane CITATION. 
Since we only use sightlines with $\glat$ between $-10^\circ$ and $+10^\circ$ and most of the low-column under-reddened gas is outside of this region, we do not currently need to worry about this effect.

The high-column sightlines with less reddening-traced than gas-traced total column are marked in red.
These sightlines pass through dense parts of molecular clouds and/or the Galactic plane within the solar circle. 
In both cases, the PPP cube underestimates the total column because a high-reddening feature nearby, be it the galactic center or the dense molecular cloud, blocks out more distant stars. 
Since the gas tracers are measured in emission rather than absorption, the PPV cube includes the more distant parts of the ISM which are missing from the PPP cube. 
Secondary effects, such as variations in dust absorption properties in dense environments, may also play a role.


The second assumption, that the distribution of velocities associated with a single voxel in the PPP cube can be accurately approximated by a narrow Gaussian, is intuitively wrong but works well-enough in practice for much of the volume covered by the PPP cube.

WHAT'S THE ARGUMENT HERE? ASSUME THAT THIS IS BASICALLY TRUE FOR A VOXEL WHICH IS "SMALL ENOUGH" IN PHYSICAL TERMS, I.E. IN DX, DY, DZ RATHER THAN DL, DB, DMU. HOW BIG CAN THIS VOXEL BE? 

TWO LIMITING CASES -- VOXEL IS SMALL ENOUGH THAT THE ASSUMPTION IS VERY GOOD, VOXEL IS BIG ENOUGH THAT THE V DISTRIBUTION CONTAINS MULTIPLE BIG PEAKS. HOW MANY DISTINCT VELOCITY THINGS ARE THERE IN A PARSEC-DEEP THING?

TWO MORE LIMITING CASES -- THE ISM AS A BUNCH OF CLUMPS WHICH MOVE A LOT RELATIVE TO EACH OTHER VS. THE ISM AS A THING WHOSE VELOCITY FIELD VARIES PRETTY SMOOTHLY FOR THE MOST PART EXCEPT FOR SOME SMALL PLACES. 

LET'S THINK ABOUT HOW THESE TWO CONCEPTS -- VOXEL SIZE AND ISM VELOCITY STRUCTURE SCALE -- INTERACT. 
TAKE THE JITTERY ISM CASE. A VOXEL WHICH IS "SMALL ENOUGH" CONTAINS ONE ``LUMP''. A VOXEL BECOMES ``TOO BIG'' WHEN IT CONTAINS MULTIPLE LUMPS WITH DIFFERENT ENOUGH VELOCITIES THAT THEIR VELOCITY GAUSSIANS DON'T ADD UP TO ~ANOTHER GAUSSIAN. SO, THEIR VELOCITIES SHOULD DIFFER BY MORE THAN SOME FRACTION OF EACH LUMP'S INTERNAL DISPERSION. 

TAKE THE SMOOTH ISM CASE. IF THE MASS DISTRIBUTION IS COMPARABLY SMOOTH, THEN THE VELOCITY DISTRIBUTION OF THE VOXEL IS A CONVOLUTION OF A GAUSSIAN WITH A WIGGLY TOPHAT. IF THE TOPHAT ISN'T ALL THAT MUCH WIDER THAN THE GAUSSIAN, I.E. IF THE CHANGE IN ROTATION CURVE + LOW FREQUENCY STREAMING MOTIONS ACROSS A VOXEL IS SOME FRACTION OF THE ``SMALL VOXEL'' VELOCITY DISPERSION, THEN THINGS ARE FINE.

DUE TO THE LIMITATIONS OF THE GSF CUBE, CASE 1 IS UNLIKELY. THE LIMITATION HERE IS THAT IF YOU'VE GOT ONE MOLECULAR CLOUD ALONG THE LINE OF SIGHT, THAT'S GENERALLY ENOUGH TO MAKE YOU INCOMPLETE TO ANY ADDITIONAL MOLECULAR CLOUDS PAST THE FIRST -- THAT FALLS UNDER THE FIRST FAILURE MODE. WHENEVER CASE 2 APPLIES, THE VELOCITY WIDTH OF THE GAUSSIANS WILL BE OVERESTIMATED. WE ACTUALLY SEE THIS OUT PAST 8 OR 10 KPC OR SO. WE DON'T CARE ABOUT THAT SORT OF DISTANCE, SINCE WE'RE NEVER EVEN REMOTELY COMPLETE OUT THERE ANYWAY.

The third assumption is that our regularization scheme is (1) informative enough to 

THIS IS A VERY IN-DEPTH EXPLORATION OF FAILURE MODES THAT I CARE ABOUT. WHICH OF THESE ARE LIKELY TO BE OF INTEREST TO ANYONE ELSE EVER? IMAGINE THAT YOU'VE JUST FINISHED GIVING YOUR TALK AND NOW YOU'RE TRYING TO EXPLAIN THE WAYS IN WHICH THINGS *COULD* BE BROKEN TO BOB. 

THERE ARE A FEW WAYS IN WHICH OUR ASSUMPTIONS PRETTY CLEARLY FAIL. 
JUST TO BE CLEAR, THE BIG ASSUMPTION IS THAT THERE'S A WAY OF CORRECTLY CRINKLING UP THE DISTANCE CUBE INTO VELOCITY SPACE SUCH THAT THE RESULT WILL LOOK LIKE THE VELOCITY MAP. 
A SECONDARY ASSUMPTION IS THAT IF THERE'S MORE THAN ONE WAY OF CRINKLING THE DISTANCE CUBE, THEN OUR REGULARIZATION TECHNIQUE IS GOOD ENOUGH THAT WE'RE PICKING ONE THAT'S *PRETTY CLOSE* TO REALITY. 

THE BIG ASSUMPTION, THAT THERE'S A WAY OF CORRECTLY CRINKLING UP THE DISTANCE CUBE, WILL PRETTY OBVIOUSLY BE WRONG IF THERE'S STUFF IN ONE CUBE THAT'S NOT IN THE OTHER CUBE AND VICE VERSA. 

Despite all of the ways in which our assumptions are clearly not always true, kinetic tomography still produces \emph{pretty good} results in the regimes we can currently test (see sections \ref{sec:KT-validation} and \ref{sec:rotation_curve}). 
The kinetic tomography solution has not yet been tested outside of the two limiting cases of the detailed velocities of the environs of HMSFRs and the azimuthally averaged velocities of most of the near half of the galaxy.
Studies attempting to apply kinetic tomography outside of those two rather narrowly defined cases will need to carefully demonstrate the robustness of their conclusions to the lengthy list of likely sources of systematic error given above.

\subsection{Catastrophic outliers}
\label{sec:discussion-catastrophic}
FOR WHATEVER REASON, THE BAD MASERS JUST REALLY DON'T LOOK VERY DIFFERENT FROM THE GOOD MASERS. 

THE ONE THING, WHICH IS HARD TO QUANTIFY BUT MAAAAAYBE SORT OF THERE IS THAT THE OUTLIERS' VELOCITIES ARE PRETTY DIFFERENT FROM THE VELOCITIES OF THEIR NEIGHBORS? 

HOW CAN WE TRY TO QUANTIFY THIS -- PICK A NUMBER OF NEAREST NEIGHBORS (SELECTED IN 3D), LOOK AT DISTRIBUTION OF STANDARD DEV OF THOSE NEIGHBORS' VELOCITIES, SEE WHERE THE OUTLIERS FALL IN IT

IF, AND THAT'S STILL AN IF, "THESE ARE DIFFERENT FROM THEIR NEIGHBORS AND THEIR NEIGHBORS LOOK LIKE THE GAS" ENDS UP BEING TRUE, HOW CAN WE INTERPRET THAT? WELL, MAYBE THE OUTLIERS ARE THE ONES WHAT GOT KICKED IN THE PROCESS OF FORMATION, SO THEIR KINEMATICS HAVE DECOUPLED FROM THE GAS.

AND ANOTHER THING -- ARE THESE SOURCES SYSTEMATICALLY _OLDER_? BECAUSE YOU'D KIND OF EXPECT THE OLD STUFF TO HAVE DECOUPLED MORE FROM THE GAS, RIGHT? MAYBE THERE'S A DIFFERENCE IN TERMS OF MASER TYPE, AS WELL?

\subsection{On the plausibility and implications of our gas streaming motions}
While the exact shapes and magnitudes of the streaming motions in our velocity map depend strongly on the assumed galactic parameters (galactocentric distance to the sun, rotation curve, solar motion relative to the rotaion curve), there is no choice of galactocentric distance, \emph{flat} rotation curve, and solar motion that can completely remove them (see Figure 4 pies).
If we limit ourselves to actually plausible galactic parameter sets (panels a, b, c of 4 pies) we see streaming motions on $\sim$ 1 kpc scales with magnitudes in excess of 15 km/sec. 
This is true even if we restrict ourselves to the nearest 5 kpc and is true for any of the three plausible parameter sets. 
These gas motions would be perfectly reasonable if the Milky Way were an obvious grand design spiral such as M83 (cite that nice paper w/the streaming motions). 
Since it is not, we should first ask if they are astrophysically plausible in a a more humble sort of spiral galaxy such as our own. 

Not unexpectedly, we will argue that these spatially extended, high magnitude motions are indeed plausible.
Within 2 kpc of the Sun, the RAVE survey has detected stellar streaming of comparable spatial extent and magnitude (S2012). They interpret these motions as evidence of the gravitaional influence of the Perseus spiral arm.
While the dynamics of stars and gas in s spiral potential will not necessarily be the same, this spiral-induced stellar streming offers a an explanation for gas streaming of an essentially appropriate magnitude.

Streaming motions 3-5 kpc away in the inner galaxy can be associated with the dynamical influence of the Galactic bar. 
This interpretation is bolstered by the similarity of our non-flat rotation curve to that of Clemens (198?), whose shape and deviation magnitude 3-5 kpc from the Galactic center can be explained by the proximity of the bar to the first quadrant tangent points (Incorrect Rotation Curve). 

Our rotation curve is in general also quite similar to many that have been measured by the GHASP survey (CITE). 
The GHASP survey used $\Halpha$ measurements to derive rotation curves for NUMBER disk galaxies with NUMBER kpc of the Milky Way with masses ranging from NUMBER to NUMBER times the mass of the Milky Way.
These rotation curves often show the same sort of $\sim 15$ km/sec, $\sim 1$ kpc velocity ``corrugations'' that we see in our rotation curve. 
The GHASP collaboration derived separate rotation curves for the advancing and receding sides of their galaxies.
Often, the central radii of the velocity corrugations are slightly different on different sides of a galaxy.
These shifts imply that these deviations from flat rotation are caused by non-axisymmetric perturbations, such as spiral arms and/or bars.

Combining these pieces of circumstantial intra- and extragalactic evidence, we can conclude that $\sim 1$ kpc-scale $\sim 15$ km/sec gas streaming motions in a non-grand design spiral galaxy can not immediately be dismissed as unphysical. 
We can also loosely associate these motions with spiral arms and the bar at large (small) and small (large) galactocentric (heliocentric) radii.

If these motions are real, what conclusions can we draw from them without fixing a galactic parameter set? One immediate implication of these streaming motions is that improving the accuracy of kinematic distances will require accounting for large-scale motions beyond the rotation curve. Junichi+ (20??), e.g., estimated the distance distortion induced by assuming a flat rotation curve in a model galaxy non-circular HMSFRs velocities comparable to those of the Reid+09 HMSFRs. 
THERE ARE IMPLICATIONS

The agreement between the gas and expanded HMFSR sample's line-of-sight velocities (see Section US WINNING) suggests that the non-circularity that the non-circularity of the HMSFR's velocities is real. 
An argument against that possibility appeared shortly after the publication of Reid+ (2009), an early analysis of an a subset of the full R+14 sample, in McMillan and Binney (2010); this argument has been repeated with some frequency in discussions of discrepancies between Galactic parameters derived from the HMSFRs and other tracers (the usual suspect and hopefully someone else too).
If the HMSFRs' motions are real and due to orbital ellipticity, the argument goes, then the implied ellipticity of their orbits is higer than that of stars that are young but already out of their birth clouds. 
Since HMSFRs are temoraly closer to collisional, and hence dynamically colder, dense gas than the collisionless, and hence dynamically hotter, young stars, this dynamical state of things is not physically plausible. 
McMillan and Binney (2010) suggest some alternatives, including a bias towards bluer line-of-sight velocities, to reduce the required orbital ellipticity and bring the velocity dispersions of the two ostensibly similar populations closer together.

The agreement between the expanded HMSFR sample's line-of-sight velocities and our (completely independently) derived map suggests that the HMSFRs deviations from flat rotation are real and not due to measurement biases.
The spatial scale and similarity to spiral- and bar-induced streaming of non-circular motions in our map also suggests a resolution to the velocity dispersion inconsistency. The HMSFRs are moving coherently, with the gas they are embedded in, rather than randomly. Once the gas and theembedded HMSFRs leave the perturbations they are currently passing through, their apparent velocity disperion as a population will presumably fall to something closer to that of the current generation of young stars.

Conversely, the McMillan and Binney argument and the agreement between the HMSFR velocities and ourmap combine into yet another piece of circumstancial evidence in favor of our streaming motions being associated with spiral- and bar-induced perturbations. 
If interpreting the HMSFR motions as stable elliptical orbits is out of the qustion, then interpreting the (if anything dynamically even colder) gas motions as stable elliptical motions must be even less acceptable.
