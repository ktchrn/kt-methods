\section{Introduction}
Many of the open problems in star formation, molecular cloud evolution, and galaxy-scale gas dynamics remain open because it has not been possible to measure the most useful instantaneous quantity for resolving them -- the vector field of 3D gas velocity as a function of 3D position, which we will call the 3-velocity field -- over an extended area of sky.
Since non-line-of-sight velocities are derived from proper motions, which cannot be reliably measured for extended sources, 3-velocity fields are likely to remain inaccessible. 
For many of these open problems, a measurement of the line-of-sight velocity as a function of 3D position, the 1-velocity field, would be enough to make non-trivial progress. 
We have developed a method for deriving the 1-velocity field of the interstellar medium (ISM) from position-position-velocity and position-position-position measurements. 


This is particularly true in our own galaxy, since a lot of the interesting motion is expected to happen in the plane.
could be all sorts of big streaming motions that would be directly detectable CITE EXAMPLES OF BIG STREAMING MOTIONS.
allows statistical approach to target smaller motions e.g. cloud rotation CITE ANGULAR MOMENTUM.

in limited, point-like cases (i.e. masers, UC HII regions), distances can be measured using parallax.
in other, also limited cases, it's possible to identify a distinct object ("a cloud") and then put bounds on the cloud's distance, typically using absorption measurements. 
if you have a PPV cube, can trade V for D (for certain $\ell$, b, values) by assuming a rotation curve. 
this assumes very regular rotation, means you can't answer how the ISM is moving around relative to the rotating frame.

to get at that sort of question, need to allow deviations from the rotation curve.
Other CfA Person and Reid have developed a thing that does something kind of like this. 
combine line-of-sight velocity, plausible peculiar velocity distribution, other information such as neighboring pixel distances, cloud sizes, etc. to get an estimate of the distance to a structure or, optimistically, voxel in a PPV cube.

because there have been large IR/optical photometric surveys, there are now also pretty good PPP cubes, which trace out the shape of the ISM in the three spatial dimensions. 
our thing is taking a PPP cube and doing the complement of what OCP\&R do with PPV cubes -- use "context clues" to figure out their velocities.
in our case, context clues are (1) PPV cubes; (2) plausible velocity ranges given a distance, rotation curve, and a peculiar velocity distribution; and (3) the velocities of neighboring PPP voxels. 
a parallel construction -- you have masses at sites in 3-space and you want to assign line-of-site velocities to them in such a way that (1) the resulting PPV cube looks like the actual PPV cube; (2) the velocities make sense given our knowledge of the rotation curve and typical deviations from it; and (3) masses that are nearby in space together will usually have similar velocities.

we need to check our answers
compare azimuthal averages to old azimuthal averages (rotation curves)
compare our velocity to the measured velocity at positions of point-like things for which precise 3-positions and velocities exist (masers)
very good agreement with point-like comparison, mostly agreement with rotation curves suggests we can probably trust the results

structure of the rest of the paper