\section{Application -- Milky Way rotation curve}
\label{sec:rotation_curve}

\subsection{Rotation curve fitting}
Since we have a measurement of the $\vlos$ component of the ISM's velocity field over a sizable chunk of the Milky Way disk, we can, in principle, derive the Galactic rotation curve.
Comparing the result to other rotation curves would serve as a check of the quality of our $\vlos$ measurement on large spatial scales, complementing Section \ref{sec:KT-validation}.
Deriving the rotation curve of the ISM from this many independent measurements would also be an advance over most earlier work, which was limited to using measurements of either geometrically special parts of the Galaxy, (e.g. \Clemens) or individual points with potentially non-uniform orbital phase coverage (e.g. \citealt{Reid:2009jb}, \Reid). 

The line-of-sight velocity $\vlos$ at some $\glon$ and heliocentric distance $d$ in the Galactic plane (meaning at $\glat=0^\circ$) is 
\begin{equation}
\label{eqn:vlos_mean}
\vlos(\glon, d) = \frac{V_\phi (\RGC) R_\odot \sin\glon}{\RGC} + 
\frac{V_R (\RGC) \left(R_\odot \cos \glon + d (1 - 2 \cos^2 \glon)  \right) }{\RGC} - V_{\phi, \odot} \sin \glon + V_{R, \odot} \cos \glon,
\end{equation}
where $V_\phi$ and $V_R$ are azimuthal (increasing clockwise) and radial (increasing outward from the Galactic center) velocities, $R_\odot$, $V_{\phi, \odot}$, and $V_{R, \odot}$ are the Galactocentric radius, azimuthal velocity, and radial velocity of the Sun, and $\RGC$ is the Galatocentric distance to the point specified by $\glon$ and $d$:
\begin{equation}
\RGC = \sqrt{R^2_\odot + d^2 - 2 R_\odot d \cos \glon}.
\end{equation}
Here, we have assumed that $V_\phi$ and $V_R$ depend only on $\RGC$, meaning that they are constant as a function of Galactic azimuth.
We will assume that there are no azimuthally constant radial flows, meaning that $V_R(\RGC) \equiv 0\text{ km/sec}$.
Following, e.g., \citet{Reid:2009jb} and \citet{Bovy_2009}, we allow for non-zero azimuthal and radial velocity dispersions, parametrized as independent normal distributions. 
In this parametrization, the line-of-sight velocity dispersion is a normal distribution with variance
\begin{equation}
\label{eqn:vlos_var}
\sigma^2_\mathrm{los} = \left(\sigma_\phi \frac{R_\odot \sin \glon}{\RGC} \right)^2 + 
\left(\sigma_R  \frac{R_\odot \cos \glon + d (1 - 2 \cos^2 \glon)}{\RGC} \right)^2,
\end{equation}
where $\sigma_\phi$ and $\sigma_R$ are the standard deviations of the azimuthal and radial velocity dispersions. 

Our ability to precisely constrain the terms in equations \ref{eqn:vlos_mean} and \ref{eqn:vlos_var} is determined by our $\glon$ and heliocentric distance ranges. 
Since we have measurements over large ranges in both \glon and heliocentric distance, we can precisely infer not only $V_\phi(\RGC)$ and but also $R_\odot$, $V_{\phi, \odot}$, and $V_{R, \odot}$. 
Our ability to infer them \emph{accurately} depends on the accuracy of the $\vlos$ measurements and the correctness of the model.
We have shown that the $\vlos$ measurements are likely to be reasonably accurate; the correctness of the model is less certain.

The biggest potential pitfall is that we have assumed departures from circular rotation, or streaming motions, are spatially uncorrelated.
The magnitude of the systematic errors this sort of model misspecification leads to is depends on the spatial scale on which the streaming motions actually are correlated.
As the scale increases from sub-pixel (i.e. independent) to of order half the extent of the measurement region to of order the extent of the measurement region or larger, the inferred parameters become less and less accurate. 
When the scale is non-negligible but still small compared to the measurement region, the dispersion widths $\sigma_\phi$ and $\sigma_R$ become unreliable.
Once the scale becomes large enough that opposite deviations from circular rotation are not statistically likely to average out, all of the other parameters -- $V_\phi(\RGC)$, $R_\odot$, $V_{\phi, \odot}$, and $V_{R, \odot}$ -- become biased.
Finally, once the scale becomes large enough that the entire measurement region is just looking at a single streaming motion, the streaming motion becomes part of the derived rotation curve.

In figure \ref{fig:six_pies}a, there are clear and obvious streaming motions that extend over several kiloparsecs. 
We can try to model this velocity field with a flat rotation curve, but we should expect the results to be quite biased.
If we attempt to derive a constant $V_\phi$, $R_\odot$, $V_{\phi, \odot}$, and $V_{R,\odot}$ from the  nearest 4 kpc of our $\vlos$ measurements, we find that all of these values all have low uncertainties but large biases relative to essentially all recent measurements of them from the literature.



