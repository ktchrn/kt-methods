\section{The rotation curve of the Milky Way and spatially coherent streaming motions}

The velocity field shown in the previous section has many, kpc-scale velocity structures, having subtracted the largest scale: a canonically flat circular velocity curve, $V_c\left(R\right) \equiv V_c$. We refer to $V_c$ as a flat rotation curve. A flat rotation curve is the first of the two common ways to describe the largest scale feature of a galactic velocity field. The second common way is to measure the azimuthally averaged azimuthal velocity of a galaxy as a function of radius. We call this a general rotation curve, $V_\phi\left(R\right)$. Some of the features we see as residuals from a flat rotation curve (e.g. Figure 4) become part of the bumps and wiggles of a general rotation curve. 

\label{sec:rotation_curve}

\subsection{The rotation curve}
\label{sec:rotation_fit}
%The Milky Way ISM's velocity field can be considered a as a composition of the Galactic rotation curve (the magnitude of the Galaxy's rotational velocity as a function of Galactic radius) and streaming motions. 
%The rotation curve contains information about the radial profile of the Galaxy's mass distribution and streaming motions can be indicative of perturbations in the Galactic potential, such as spiral arms and the bar.
%Since there are many derivations of the Galactic rotation curve in the literature, we can also use it as a check of the accuracy of our $\vlos$ map on large spatial scales, complementing the previous section's validation on small spatial scales.

%We measure the Galactic rotation curve by comparing the observed $\vlos(\glon, d)$ to line-of-sight velocities corresponding to different combinations of rotation curves and Galactic parameters in the Galactic plane (meaning at $\glat=0^\circ$): 
%The line-of-sight velocity $\vlos$ at a Galactic longitude $\glon$ and heliocentric distance $d$ in the Galactic plane (meaning at $\glat=0^\circ$) is 
The velocity field of a component of the Galaxy is a superposition of the Galactic circular velocity curve and that component's streaming motions. 
The circular velocity curve is determined entirely by the radial profile of the Galactic mass distribution, while streaming motions are caused by perturbations in the Galactic potential and dynamical effects, whose specifics can vary from Galactic component to Galactic component. 
To distinguish the theoretical quantity, the circular velocity curve, from the empirical quantity, the azimuthal average of the azimuthal component of the velocity field, we will call the later a \emph{rotation curve}. 
This section is dedicated to measuring the rotation curve of the Milky Way's ISM from the KT-derived $\vlos(\glon, \glat, d)$ 3-cube and comparing the results to other Milky Way rotation curves. 
We will discuss streaming motions and the relation between the Milky Way ISM's rotation curve and circular velocity curve in Section \ref{sec:rotation_discussion}.

We assume that the global motion of the Milky Way ISM can be entirely described by a rotation curve, $V(\RGC)$. 
In particular, we assume that there are no global radial flows and that azimuthal and radial departures from the rotation curve average to zero over the Galaxy.
We allow for local, spatially independent departures from the rotation curve and model these departures as a Gaussian perturbation superimposed on the global motion. 
We assume that the azimuthal and radial components of the velocity perturbation are uncorrelated and have standard deviations $\sigma_\phi$ and $\sigma_R$.

In addition to the motion of the ISM, we need to describe the location and motion of our observing site, the Sun. 
We use $\RGC_{, \odot}$ for the Sun's Galactocentric radius and $V_{\phi, \odot}$ and $V_{R, \odot}$ for its azimuthal and radial motion relative to the Galactic center. 
Because we will only be using the $\glat \approx 0 \deg$ portion of the $\vlos(\glon, \glat, d)$ cube, we can ignore the Sun's motion perpendicular to the Galactic plane.

To compute the likelihood of a parameter set $V_\phi (\RGC)$, $\sigma_\phi$, $\sigma_R$, $R_{\odot}$, $V_{\phi, \odot}$, and $V_{R, \odot}$, we first need to convert these parameters to a model $\hat{\vlos}(\glon, d)$. 
The Galactocentric radius of a given $\glon$ and $d$ at $\glat = 0 \deg$ is 
\begin{equation}
\RGC = \sqrt{\RGC^2_{,\odot} + d^2 - 2 \RGC_{, \odot} d \cos \glon}.
\end{equation}
The mean line-of-sight velocity at a $\glon$ and $d$ is
\begin{equation}
\label{eqn:vlos_mean}
\hat{\vlos}(\glon, d) = \frac{V_\phi (\RGC) \RGC_{,\odot} \sin\glon}{\RGC}  - V_{\phi, \odot} \sin \glon + V_{R, \odot} \cos \glon.
\end{equation}
The standard deviation of the distribution of line-of-sight velocities about this mean is 
\begin{equation}
\label{eqn:vlos_var}
\sigma^2_\mathrm{los} = \sqrt{\left(\sigma_\phi \frac{\RGC_{,\odot} \sin \glon}{\RGC} \right)^2 + 
\left(\sigma_R  \frac{\RGC_{,\odot} \cos \glon + d (1 - 2 \cos^2 \glon)}{\RGC} \right)^2}.
\end{equation}
To these model $\hat{\vlos}(\glon, d)$ values we compare the mass-weighted $\glat$-average of the KT-derived $\vlos(\glon, \glat, d)$ cube for $-2.5 < \glat < + 2.5$; we will call this $\glat-$averaged quantity the $\vlos(\glon, 0 \deg, d)$ \emph{map}.
Our likelihood function for a single $(\glon, 0\deg, d)$ voxel is a Gaussian with mean $\hat{\vlos}(\glon, d)$ and standard deviation $\sigma_\mathrm{los}(\glon, d)$.
The full likelihood function is the product of the single-voxel likelihood functions over all $(\glon, 0\deg, d)$ voxels.

We fit two sets of models, one in which we assume a flat rotation curve and let the Solar parameters vary and one in which we assume a piecewise-linear rotation curve with breaks every 0.25 kpc and fix the solar parameters to $\RGC=8.1$ kpc CITEP E.G. BOVY, $v_{\phi, \odot}=TK$, and $v_{R, \odot}=TK$ (CITE THAT STANDARDS PAPER THAT'S CITED IN IBID).
We use different portions of the $\vlos(\glon, 0\deg, d)$ map for the different models. 
There are three competing considerations for choosing what portion of the map to use --- the quality of the KT solution drops beyond some maximum heliocentric distance, model parameters become less degenerate as the $\glon$ and $\RGC$ ranges of the data widen, and parts of the Galaxy are known to deviate from flat rotation (e.g. CITEALT WRONG ROTATION CURVE). 
Motivated by the third consideration, our fiducial ranges for the flat rotation curve case are TK and TK. 
In the piecewise-linear rotation curve case, the first and second considerations are more important, so we use DRANGE TK and RGCRANGE TK, which translates to a non-separable $\glon$-$d$ constraint.

For the fiducial $\glon$- and $d$-ranges, our results for the flat rotation curve model are $V_\phi=TK \pm TK$, $\sigma_\phi = TK \pm TK$, $\sigma_R = TK \pm TK$, $R_\odot = TK \pm TK$, $v_{\phi, \odot} = TK \pm TK$, and $v_{R, \odot} = TK \pm TK$.
The $V_\phi$ and $R_{\odot}$ values are in tension with the IAU-recommended values, as well as the results of CITET BOVY and CITET REID, if we take our quoted uncertanties to be accurate. 
If we vary the outer limit of the $d$-range used in the fit from TK to TK, $V_\phi$ varies from TK to TK and $R_\odot$ varies from TK to TK. 
This variation is smooth as a function of the outer limit of the $d$-range, indicating that the variation is not driven by a small problem region.
We believe that the cause of this variation is global model misspecification --- we have assumed that deviations from the rotation curve are spatially uncorrelated when they are in fact correlated, and on spatial scales that are of order the spatial extent of the $\vlos(\glon, 0\deg, d)$ map.
We will discuss this issue further in Section \ref{sec:rotation_discussion}. 

Our best-fit piecewise-linear rotation curve is shown, along with the CITET CLEMENS rotation curve for comparison, in Figure \ref{fig:rotation_curves}. 
Both curves can be qualitatively described as 3 ``bumps'' or ``wiggles'' about a mean circular velocity. 
Pointwise, the two curves agree to within about 10 km/sec. 
In terms of the qualitative description, the curves are in even better agreement --- the mean circular velocities and the locations of the bumps in the two curves are essentially the same, while the amplitudes of the bumps are greater by the same factor of $\approx 10$ km/sec for the KT-derived rotation curve.
The differences that are present can most likely be explained by differences in the sets of $\vlos$ measurements that go in to curves' derivations. 

%where $V_\phi$ and $V_R$ are azimuthal (increasing clockwise) and radial (increasing outward from the Galactic center) velocities, $R_\odot$, $V_{\phi, \odot}$, and $V_{R, \odot}$ are the Galactocentric radius, azimuthal velocity, and radial velocity of the Sun, respectively, and $\RGC$ is the Galatocentric distance to the point specified by $\glon$ and $d$:

%Here, we have assumed that $V_\phi$ and $V_R$ depend only on $\RGC$, meaning that they are constant as a function of Galactic azimuth.
%We will assume that there are no azimuthally constant radial flows, meaning that $V_R(\RGC) \equiv 0\text{ km/sec}$.
%Following, e.g., \citet{Reid:2009jb} and \citet{Bovy_2009}, we allow for non-zero azimuthal and radial velocity dispersions, parametrized as independent normal distributions. 
%In this parametrization, the line-of-sight velocity dispersion at a single location is a normal distribution with variance
%\begin{equation}
%\label{eqn:vlos_var}
%\sigma^2_\mathrm{los} = \left(\sigma_\phi \frac{R_\odot \sin \glon}{\RGC} \right)^2 + 
%\left(\sigma_R  \frac{R_\odot \cos \glon + d (1 - 2 \cos^2 \glon)}{\RGC} \right)^2,
%\end{equation}
%where $\sigma_\phi$ and $\sigma_R$ are the standard deviations of the azimuthal and radial velocity dispersions. 

%We use two models for the rotation curve and other Galactic parameters. 
%In the first case, we assume a flat rotation curve, expressed as a single radially constant $V_\phi(\RGC) \equiv V_\phi$, and solve for the best-fit $V_\phi$, $V_{\phi, \odot}$, $V_{R, \odot}$, $\RGC$, $\sigma_\phi$, and $\sigma_R$. 
%In the second, we assume a continuous, piecewise linear rotation curve and fix the Galactic parameters to a fiducial set of values -- $\RGC = 8.1\text{ kpc}$, $V_{\phi, \odot} - V_{\phi} = 15.3 \text{ km/sec}$, and $V_{R, \odot} = - 10.3 \text{ km/sec}$. 
%In the first case, we only use $\vlos$ measurements within the nearest 4 kpc.
%This avoids the inner 4 kpc of the Galaxy, where the rotation curve is known to be non-flat.
%In the second case, we use $\vlos$ measurements that are between 2 and 15 kpc from the Galactic center on the near side of the Galaxy. 
%The breakpoints of the piecewise linear curve are placed every 0.25 kpc.

%Our results for the parameters of the first model are $\RGC=8.90 \pm 0.04 \text{ kpc}$, $V_\phi = 279 \pm 1.5 \text{ km/sec}$, $V_{\phi, \odot} - V_\phi = 28.5 \pm 0.1 \text{ km/sec}$, and $V_{R, \odot} = 6.2 \pm 0.1 \text{ km/sec}$.
%While the formal uncertainties on these values are small, we can tell that the systematic uncertainties must be large, particularly on $\RGC$ and $V_\phi$, because the results depend strongly on which portion of the measurements we include in the fit.
%The parameter values we give above use the measurements from the nearest 4 kpc of the Galactic disk.
%If we instead use measurements from the nearest 3 kpc, we find that $\RGC$ changes to $8.05 \pm 0.06 \text{ kpc}$ and $V_\phi$ remains unchanged, while if we use measurements from the nearest 6 kpc and exclude the inner Galaxy by only using sightlines with $\glon > 45^\circ$, we get $\RGC = 7.43 \pm 0.03 \text{ kpc}$ and $V_\phi = 223 \pm 1 \text{ km/sec}$.
%We believe the cause of these large changes in parameter values is that as we vary the region being analyzed, we include and exclude different large-scale streaming motions. 
%Since these large-scale streaming motions are not included in the model specification, they end up dragging parameters in different directions.
%We discuss this further in Section \ref{sec:rotation_discussion}.

%The piecewise-linear rotation curve is shown in Figure \ref{fig:rotation_curves}, with the \Clemens rotation curve for comparison. 
%These two non-flat rotation curves mostly agree, in the sense that deviations from flat rotation have the same sign, if not always the same magnitude.

\subsection{Spatially coherent streaming motions}
\label{sec:rotation_discussion}
In Figure \ref{fig:six_pies}, we show residuals between the $\vlos$ measurements and (panel b) the flat rotation curve derived from the nearest 4 kpc and (panel c) the piecewise-linear rotation curve. 
For comparison, we also show residuals between the $\vlos$ measurements and (panel a) our fiducial flat rotation curve and Galactic parameters, (panel d) the \Clemens non-flat rotation curve, (panel e) the \cite{2012ApJ...759..131B} flat rotation curve, (panel f) and the \Reid A5 model flat rotation curve.
In every panel, there are obvious, multi-kpc regions with consistent non-zero velocity residuals.
Most of these regions persist across all six panels, though there is some panel-to-panel variation in the exact bounds and residual magnitude of each region. 
For example, there is always a blue-shifted feature just outside the solar circle at $90^\circ \lesssim \glon \lesssim 180^\circ$, a smaller red-shifted feature $\approx 12$ kpc from the Galactic center at the high-$\glon$ boundary of the map, and another red-shifted feature just inside the solar circle. 

Persistence across different sets of rotation curves and Galactic parameters provides evidence that the spatially coherent residuals are not caused by a misspecification of the global Galactic motion and geometry.
We can argue for the accuracy of most of the residual regions by taking the HMSFR measurements as truth and appealing to the regions' strong spatial coherence.
The argument goes as follows -- if the $\vlos$ map accurately predicts the velocities of HMSFRs that are spatially coincident with a specific residual region, it is reasonable to assume that the $\vlos$ values assigned to the rest of that region are also likely to be accurate. 
Based on the discussion in Section \ref{sec:KT-validation}, we can claim that the HMSFR velocities are accurately predicted everywhere outside of the inner 4-5 kpc of the Galaxy or so.
A visual inspection of Figure \ref{fig:maser_pie} shows that there are HMSFRs in every residual region within about 6 kpc of the Sun. 
We interpret the combination of these two facts as evidence that all of the residual regions shown in Figures \ref{fig:maser_pie} and \ref{fig:six_pies} outside of the inner 4-5 kpc of the galaxy are, in fact, actually present in the velocity field of the Milky Way ISM. 

Once we have established that the spatially-structured residuals are real, we can ask what they are.
In particular, we can ask what fraction of the residuals can be ascribed to radial rotation curve variations and what fraction should instead be interpreted as streaming motions. 
If we compare the residuals from the two radially-varying rotation curves in Figure \ref{fig:six_pies} (panels c and d) to the residuals from the four flat rotation curves (panels a, b, e, and f), the typical magnitudes of the former are clearly lower than typical magnitudes of the latter. 
While we could interpret this difference in residual magnitudes as evidence that the residuals are mostly attributable to radial rotation curve variations, that interpretation would only be true for a very empirical definition of a rotation curve.

We can consider two definitions of a rotation curve: the azimuthal average of the azimuthal component of the Galactic velocity field (empirical) and the circular velocity in the Galactic gravitational potential (theoretical), both of which may be functions of Galactocentric radius. 
If azimuthal departures from the theoretical rotation curve, i.e. azimuthal streaming motions, at some Galactocentric radius do not average to zero, then the empirical and theoretical rotation curves at that radius will not be equal. 
We are particularly prone to cross-talk between spatially coherent streaming motions and radial variations in the rotation curve because of the limited azimuthal coverage of our $\vlos$ measurements.
At some radii, a streaming motion that is coherent over a few kiloparsecs is enough to fill most of the azimuth range in our map at those radii. 
In the inner Galaxy, particularly near the tip of the Galactic bar, this danger has been pointed out by e.g. \citet{Chemin_2015}.
Those authors argue that most of the radial variations in the inner Galaxy in \Clemens, in particular, are actually streaming motions.
While our azimuth coverage is wider than that of \Clemens, the similarity in the rotation curve shapes (Figure rotation curves) suggests that the point still applies.

While \citet{Chemin_2015} assume that deriving the rotation curve in the outer Galaxy should be relatively free of hassle, extragalactic studies such as \citet{Meidt_2013} and (some of those Halpha ones) have shown that strongly spiral galaxies can have quite strong streaming motions far away from the bar. 
So, the advice seems to be to just steer clear of assuming fluctuations in azimuthal velocities have much to do with the theoretical rotation curve -- it's probably all streaming motions. 