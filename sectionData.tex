\section{Data}
\label{sec:data}
\subsection{HI and CO cubes}

Radio emission lines in HI and CO trace the two dominant constituents of the Galctic ISM, atomic and molecular gas. Ionized phases of the ISM do not contribute significantly to the column density, and will therefore not trace the extinction measured in \cite{Green_2015}. 21-cm line emission from the hyperfine transition of HI is usually optically thin and its integral is an excellent tracer of HI column.
\begin{equation}
N_{HI} = 1.8 \times 10^{18} \rm cm^{-2} \frac{1}{\rm K~km/s} \int T_B dv
\end{equation}
In some cases the 21-cm line can become optically thick, and thus an underestimate of the HI column, but largely in H$_2$ dominated regions \cite{Goldsmith_2007}. 

We trace molecular gas here using the 115 GHz 1-0 rotational transition of CO. The integral of this emission line can be converted to total hydrogen column CITE using

\begin{equation}
X_{CO} = .
\end{equation}

This conversion factor has number of known weaknesses, stemming from much more complex excitation and opacity effects, as well as real variation in relative population of CO and $H_2$ molecules. We will address the imapacts of these weakenesses in \S \ref{sec:validation}. 


sources and regridding process
post-processing, which includes multiplying by constants to get H nucleon counts and median filtering in lb space to get rid of single voxel artifacts in the CO map

\subsection{Reddening cube}
source, regridding process?
brief rundown of how these are made
post-processing, aka multiplying by a constant

\subsection{Reid masers}
final source
processing to get to distances (just a conversion from parallax) and line-of-sight velocities
how many/which ones we exclude due to lack of overlap with reddening cube