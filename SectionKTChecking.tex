\subsection{Checking the KT solution}
\label{sec:KT-validation}

We have used the unregularized and regularized versions of the kinetic tomography technique to derive PPPV 4-cubes for the region of the galactic plane with $0^\circ < \ell < 220^\circ$ and $-10^\circ < b < 10^\circ$. %We will refer to these two solutions by the names of their methods, i.e. as the UKT and RKT solutions.
These solutions can be thought of as either one 4-cube of the ISM density in PPPV space or as a set of three 3-cubes of ISM density, central velocity ($v_{los}$), and velocity dispersion in PPP space. One benefit of the ``three 3-cubes'' representation is that the accuracy of each of the three 3-cubes can be checked separately. The accuracy of the 3-cube of ISM density map is discussed in Schlafly+. In this section, we check the accuracy of the unregularized and regularized $v_{los}$ 3-cubes against the R14 dataset, which is described in section \ref{sec:data-HMSFR}. We find that both $v_{los}$ 3-cubes represent clear improvements over flat and \cite{Clemens_1985} rotation curves, and that the regularized solution is statistically consistent with the R14 measurements, with the exception of a handful of outliers. 

The quantities we are comparing are the line-of-sight velocity of high-mass star forming regions (HMSFR) from R14 with the $v_{los}$ 3-cube evaluated at the $\ell$, $b$, and distance of the HMSFRs. To compute the central velocity from the $v_{los}$ 3-cube, we take the $v_{los}$ profile of the $\ell$, $b$ cell that contains a HMSFR, linearly interpolate $v_{los}$ between the centers of the distance bins, and evaluate this piecewise linear $v_{los}$ profile at the distance of the HMSFR given by R14. We estimate the uncertainty of this velocity by propagating the distance uncertainty of the HMSFR to a $v_{los}$ uncertainty. This distance-derived uncertainty is added in quadrature with the radial velocity uncertainty quoted for the HMSFR in R14. The typical magnitude of this total uncertainty is 3-10 km/sec; Figure (HMSFR on profile) shows an HMSFR for which this uncertainty is particularly large. We also compute the $v_{los}$ predicted for each HMSFR by the \citet{Clemens_1985} rotation curve using the same procedure; we refer to these $v_{los}$ values as the \Clemens values. 

Figure (maser comp) compares the R14 measured HMSFR $v_{los}$ to the \Clemens, unregularized, and regularized $v_{los}$ values. There is a clear improvement in agreement from the \Clemnens to the unregularized, and from the unregularized to the regularized $v_{los}$. The unregularized $v_{los}$ captures the full amplitude of the discrepancy between the R14 measurements and a flat rotation curve while \Clemens rotation curve does not. 

For a more quantitative assessment, we can analyze the distribution of \emph{standardized residuals} -- residuals divided by uncertainties. This distribution is shown in Panel (a) of Figure (5). For comparison, the figure also shows a normal distribution with mean zero and standard deviation one, a \emph{standard normal} distribution. If the uncertainties are accurate and errors are normally distributed, the standardized residuals should be a drawn from the standard normal distribution. Of the 99 HMSFRs that fall within our $\ell$ and $b$ range, 5 have residuals that are more than three standard deviations away from zero; we discuss these six in section \ref{sec:discussion-catastrophic}. The standardized residuals of the remaining 94 objects are consistent with being drawn from the standard normal distribution.  We discuss what this means for the more global accuracy of the regularized solution in section \ref{sec:discussion-systematics}. 